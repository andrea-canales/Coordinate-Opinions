 \documentclass[letterpaper,11pt]{article}
\usepackage{tgpagella}
\usepackage[utf8]{inputenc}
\usepackage{graphicx}
\usepackage{fullpage,paralist}
\usepackage{amsmath, amssymb, amsthm}
\usepackage{comment,hyperref}
\usepackage{thmtools}
\usepackage{tikz}
\usepackage{pgfplots}
\usepackage{varwidth}
\pgfplotsset{compat=1.10}
\usepgfplotslibrary{fillbetween}
\usetikzlibrary{backgrounds}
\usetikzlibrary{patterns}

% Special commands
\newcommand{\PP}{\mathbb{P}}
\newcommand{\calR}{\mathcal{R}}
\newcommand{\calG}{\mathcal{G}}
\newcommand{\calH}{\mathcal{H}}
\newcommand{\calM}{\mathcal{M}}
\newcommand{\calL}{\mathcal{L}}
\newcommand{\calP}{\mathcal{P}}
\newcommand{\calQ}{\mathcal{Q}}
\newcommand{\calF}{\mathcal{F}}
\newcommand{\calC}{\mathcal{C}}
\newcommand{\CC}{\mathcal{C}}
\newcommand{\gumbel}{\calF_{\text{g}}}
\newcommand{\frechet}{\calF_{\text{f}}}
\newcommand{\mhr}{\text{MHR}}
\newcommand{\calA}{\mathcal{A}}
\newcommand{\calT}{\mathcal{T}}
\newcommand{\calE}{\mathcal{E}}
\newcommand{\RR}{\mathsf{R}}
\newcommand{\NN}{\mathbb{N}}
\newcommand{\ZZ}{\mathbb{Z}}
\newcommand{\FF}{\mathbb{F}}
\newcommand{\GL}{\text{GL}}
\newcommand{\EE}{\mathsf{E}}
\newcommand{\ass}{\text{Assign}}
\newcommand{\clp}{\text{clp}}
\newcommand{\config}{\text{conf}}
\newcommand{\conf}{\text{conf}}
\newcommand{\apx}{\text{apx}}
\newcommand{\ALG}{\mathrm{ALG}}
\newcommand{\OPT}{\mathrm{OPT}}
%further commands
\newcommand{\sG}{\mathsf{G}}
\newcommand{\fr}{\mathsf{f}}
\newcommand{\fsf}{\mathsf{F}}
\newcommand{\gsf}{\mathsf{G}}
\newcommand{\hsf}{\mathsf{H}}
\newcommand{\ksf}{\mathsf{K}}
\newcommand{\asf}{\mathsf{A}}
\newcommand{\bsf}{\mathsf{B}}
\newcommand{\csf}{\mathsf{C}}
\newcommand{\dsf}{\mathsf{D}}
\newcommand{\esf}{\mathsf{E}}
\newcommand{\lsf}{\mathsf{L}}
\newcommand{\nsf}{\mathsf{N}}
\newcommand{\ssf}{\mathsf{S}}
\newcommand{\rsf}{\mathsf{R}}
\newcommand{\tsf}{\mathsf{T}}
\newcommand{\usf}{\mathsf{U}}
\newcommand{\vsf}{\mathsf{V}}
\newcommand{\wsf}{\mathsf{W}}
\newcommand{\isf}{\mathsf{I}}
\newcommand{\omsf}{\mathsf{\Omega}}
\newcommand{\cc}{\mathsf{Cp}}
\newcommand{\rk}{\mathsf{rk}}
\newcommand{\cost}{\mathsf{cost}}
\newcommand{\degsf}{\mathsf{deg}}
\newcommand{\tv}{\mathsf{TV}}
\newcommand{\tmix}{t_{\mathsf{mix}}}
\newcommand{\psf}{\mathsf{P}}
\newcommand{\xsf}{\mathsf{x}}
\newcommand{\poi}{\mathsf{Geom}}
% Environments
\newtheorem{thm}{Theorem}
\newtheorem{theorem}{Theorem}
\newtheorem{lemma}{Lemma}
\newtheorem{corollary}{Corollary}
\newtheorem{remark}{Remark}
\newtheorem{prop}[thm]{Proposition}
\newtheorem{definition}{Definition}
\newtheorem{example}{Example}
\newtheorem{proposition}{Proposition}
\newtheorem{claim}{Claim}
\newtheorem{conjecture}{Conjecture}
\newtheorem{intuition}{Intuition}

% Comments
%\setlength{\marginparwidth}{1in}
\usepackage[textsize=tiny,textwidth=2cm]{todonotes}
\newcommand{\accom}[1]{\todo[color=blue!25!white]{Andrea: #1}}
\newcommand{\vvcom}[1]{\todo[color=red!25!white]{Victor: #1}}

%Algorithms
\usepackage{multirow}
\usepackage[noend]{algpseudocode}
\usepackage{algorithm,algorithmicx}
\usepackage{xcolor}
\def\NoNumber#1{{\def\alglinenumber##1{}\State #1}\addtocounterhh{ALG@line}{-1}}
\newlength{\algofontsize}
\setlength{\algofontsize}{6pt}
\hypersetup{
	colorlinks,
	linkcolor={red!50!black},
	citecolor={blue!50!black},
	urlcolor={blue!80!black}
}

\begin{document}
	\algrenewcommand\algorithmicrequire{\textbf{Input:}}
	\algrenewcommand\algorithmicensure{\textbf{Output:}}
	
	\title{Sustainability of Opinion Coordination in Social Networks
%	 \thanks{This work was partially supported by blablabla}
	 }
	\author{Andrea Canales
	\thanks{Institute of Social Sciences, Universidad de O'Higgins. {\tt andrea.canales@uoh.cl}}
	\and Victor Verdugo			
	\thanks{Department of Mathematics, London School of Economics and Political Science. {\tt v.verdugo@lse.ac.uk}}
	\thanks{Institute of Engineering Sciences, Universidad de O'Higgins. {\tt victor.verdugo@uoh.cl}}
	}
\date{\vspace{-1em}}
\maketitle
\begin{abstract}
The process by which opinions spread through a large social network can be modeled as an interaction between the initial opinion of an agent and their neighbor's opinions. This process can be affected by stubborn agents, who maintain their initial opinion fixed. We characterize under which conditions a group of regular agents can coordinate their opinions and behave like stubborn agents. We show that if an agent has incentives to coordinate his opinion in the first period, he announces the opinion of the coordination in every period. There exist an interval of opinions that can be sustained by the coordinated agents, which   depends on their initial opinion and the connectivity of this subgraph. We also study how this coordination impact the convergence speed  and characterize the convergence time .
	\accom{Todos los cambios son bienvenidos y sorry el ingles}

	\vvcom{Otra opcion de titulo es {\it Robust Sustainability of Beliefs Coordination}}
\end{abstract}
\newpage

\tableofcontents

\section{Introduction}
\accom{Hacer la analogía de este modelo a la literatura de carteles incompletos. No todos pueden escoger ``precios'',solo los invitados a coordinarse y buscamos condiciones bajo las cuales esta coordinación  es maximal}
\section{Local Interaction and Coordination Model}

%\noindent{\it Coordinations.} 
Consider a graph $G=(V,E)$ representing a social network.
Every node $v\in V$ in the network has an intrinsic opinion $\gamma_0(v)$.
At every time step $t\in \NN$ and for every $u\in V$, $\gamma_t(u)$ denotes the opinion {\it declared} by $u$.
We say that $(C,\beta)$ is a {\it coordination} if every node $v\in C$ declares the same opinion $\beta$, that is, for every $t\in \ZZ_+$ and for every $u\in C$, $\gamma_t(u)=\beta$.
Every node $u\notin C$ updates the opinion according to 
%\begin{equation}
%\gamma_t(v)=\text{argmin}\Big\{\text{var}_v(\alpha_C,\gamma_{t-1}):\alpha_C\in \RR\Big\}.
%\end{equation}
%Every node $v\in V\setminus C$ declares an opinion according to the following opinion update,
%For every $t\in \mathbb{Z}_+$, the opinion of node $v$ updates according to the DeGroot model, that is, 
\begin{align}
\label{eq:update}
\gamma_{t+1}(u)&=\frac{\eta_u}{\degsf(u)+\eta_u}\cdot \gamma_0(u)+\frac{1}{\degsf(u)+\eta_u}\sum_{v\in \nsf(u)}\gamma_{t}(v),
%		&=(1-\eta)\cdot \gamma_{t-1}(v)+\frac{\eta}{|N(v)\cap R|}\sum_{u\in N(v)\cap R}\gamma_{t-1}(u)+\eta \cdot \gamma_{t-1}(C),
\end{align}
%namely, it averages with a weight of $\eta$ the current opinion and with weight $1-\eta$ the average of the current opinions of the neighbors of $v$. 
For every $u\in V$, the parameter $\eta_u$ captures how resistant is $u$ to modify his initial opinion $\gamma_0(u)$.
Observe that if $C=\emptyset$ we then recover the classic Friedkin and Johnsen opinion dynamics model~\cite{}.
In what follows, it will be useful to consider the dynamics in a matrix form.
Consider the matrix $\asf\in \RR^{V\times V}$ defined as follows: 
Let $\asf(u,v)=1/(\degsf(u)+\eta_u)$ if $\{u,v\}\in E$ and $u\in V\setminus C$, and zero otherwise, and let $\bsf\in \RR^{V\times V}$ the diagonal matrix given by $\bsf(u,u)=\eta_u/(\degsf(u)+\eta_u)$ for every $u\in V\setminus C$.
Then, the dynamics of the declared opinions (\ref{eq:update}) can be written as
\begin{equation}
\label{eq:update-matrix}
\gamma_{t+1}=\asf \gamma_t+\beta \isf_C+\bsf\gamma_0,
%=\asf^{t} \gamma_1+\sum_{j=0}^{t-1}\asf^j \bsf \gamma_0+\beta\isf_C,
\end{equation}
where $\isf_C\in \RR^{V\times V}$ is such that $\isf_C(u,u)=1$ if $u\in C$, and zero otherwise.
In particular, observe that from $t=1$ every member of the coordination declares the opinion $\beta$, that is, for every $u\in C$ we have $\gamma_1(u)=\beta$.
In the following, we call $\gamma$ the {\it coordination dynamics} for $(C,\beta)$.
A similar dynamics was considered by Ghaderi and Srikant~\cite{}, to study networks with {\it stubborn} agents. 
We later provide a random walk interpretation of the above dynamics to study the evolution and long-run behavior of the process. 
%For notational convenience, in what follows we call $\gamma_0^{\beta}\in \RR^V$ the vector such that $\gamma_0^{\beta}(u)=\beta$ if $u\in C$ and $\gamma_0^{\beta}(u)=\gamma_0(u)$ if $u\in V\setminus C$.
\begin{example}
\vvcom{ejemplito con grafo chico y una o dos iteraciones}
\end{example}

%\begin{lemma}
%limiting behavior of the dynamics
%\end{lemma}
%The opinion at time $t$ is then the expected opinion of the process above.
%\vvcom{chequear esto, ajustar definicion del costo de ser necesario}
%It can be shown that the dynamics above converge to an opinion vector $\alpha_C^*$ ....{\color{red} ver OpinionSurvey2017, chapter 2.1}\\
%\vvcom{dar el background que sea necesario en esta parte, el paper EC19naive trata el modelo de degroot modificado}

%\vvcom{podemos usar Colluted en vez de Coordinated, o block, etc, no se que sera mejor}\\

\begin{comment}
\noindent{\it Sustainability.} Every node $u\in V$ faces a discount factor of $\delta\in (0,1)$, representing the {\color{red} dar la intuicion aca.}
%In the following assume that $\gamma_0(u)$ is distributed uniformly at random in $[0,1]$ for every $u\in V$, independently from each other. 
We say that a triplet $(C,\beta,\gamma_0)$ is a {\it sustainable coordination} if for every $u\in C$ we have that
\begin{equation}
\sum_{t=1}^{\infty}\delta^{t}\cost_u(\gamma_{t})\le \sum_{t=1}^{\infty}\delta^{t}\cost_u(\tilde \gamma_{t}),
\end{equation}
where $\tilde \gamma$ is the coordination dynamics for $(C\setminus \{u\},\beta)$.
That is, the expected discounted cost that faces a node $u\in C$ is less than the expected discounted cost faced by not joining the coordination.
\end{comment}
%The expectation is taken over the randomness of the intrinsic opinions $\gamma_0$.

\section{Existence and Behavior of the Long-run Opinions}
\label{sec:auxiliar}
In what follows we use a result by Ghaderi \& Srikant showing the existence and providing a characterization of the long-run opinions of a network under the presence of stubborn agents. 
%An agent is {\it stubborn} if $\eta_u>0$ and {\it fully stubborn} if $\eta_u=\infty$.
To study the dynamics, the authors construct a random walk in an auxiliary graph, and characterize the long-run opinion by studying the stationary distribution of this auxiliary random walk.
Observe that in our model, for a coordination $(C,\beta)$ we have that every $u\in C$ can be seen a fully stubborn agent for the dynamics from $t=1$.
Nevertheless, the value of $\beta$ can be chosen after the realization of each of the intrinsic opinions of the members in the coordination.
For the sake of completeness, we include the random walk construction here and state the results in our context.\\

\noindent{\it An auxiliary random walk.} Given a graph $G$ and $C\subseteq V$, consider the auxiliar graph, $\cal{G_C}=(\cal{V}_C,\cal{E}_C)$ defined as follows. 
We have $\mathcal{V}_C=V\cup T_{V\setminus C}$ where $T_{V\setminus C}=\{x_v:v\in V\setminus C\}$, that is, the set of nodes $\cal{V}_C$ consists of every node in the original graph and a copy of each node not in $C$. 
Furthermore, in the auxiliary graph every node in $V\setminus C$ is connected to its copy, that is, consider 
\begin{equation*}
\mathcal{E}_C=E\cup \Big\{\{v,x_v\}:v\in V\setminus C\Big\}.
\end{equation*}
Consider the random walk $(\xsf_t)_{t\in \ZZ_+}$ over the graph $\mathcal{G}$ with $\psf\in [0,1]^{\mathcal{V}\times \mathcal{V}}$ given by
\[
\psf(u,v)=
\begin{cases}
1/\degsf(u) & \text{ for every }u\in C\text{ and every }v\in \nsf(u),\\
1/(\degsf(u)+\eta_u)& \text{ for every }u\in V\setminus C\text{ and every }v\in \nsf(u),\\
\eta_u/(\degsf(u)+\eta_u) & \text{ for every }u\in V\setminus C\text{ and }v=x_u,\\
1 & \text{ for every }v\in V\text{ and }u=x_v,
\end{cases}
\]
For every $v\in \mathcal{V}_C$, consider $\tau_v=\inf\{t\in \ZZ_+:\xsf_t=v\}$, that is, the hitting time of vertex $v$, and let 
\begin{equation*}
\tau_C=\inf\Big\{t\in \ZZ_+:\xsf_t\in C\cup T_{V\setminus C}\Big\}
\end{equation*}
be the hitting time of the set of nodes in $C\cup T_{V\setminus C}$.
%For every $v\in V$, l
For every $v\in V$, let $\alpha_{v,C}\in \RR^{V}$ be the vector such that for every $w\in V$ we have $\alpha_{v,C}(w)=\PP_v(\tau_{x_w}=\tau)$. 
In particular, $\alpha_{v,C}$ is a probability distribution.
%when $v\in V\setminus C$, and $\alpha_C(u,v)=\Theta_u(C)$ when $v\in C$, where 
Furthermore, consider the quantity given by
\begin{equation*}
\Theta_{u,C}=\sum_{v\in C}\PP_u(\tau_C=\tau_{v}).
\end{equation*}
The value above corresponds to the probability that the random walk $(\xsf_t)_{t\in \ZZ_+}$ starting at $u\in V$ hits $C\cup T_{V\setminus C}$ in a vertex that belongs to $C$.
We now state the technical lemma from~\cite{} adapted to our context.
We include a proof of it in the Appendix. 
%\vvcom{quizás escribir lo que viene en terminos de la distribucion estacionaria $\pi$ del RW}
\begin{lemma}[\cite{GS12}]
\label{lem:limit-thm}
Let $G=(V,E)$ be a connected graph and $\gamma_0$ a vector of intrinsic opinions. 
Then, for every coordination $(C,\beta)$ there exists $\gamma_{\infty}\in \RR^V$ such that $\displaystyle\lim_{t\to \infty}\gamma_t=\gamma_{\infty}$.
Furthermore, we have that for every $v\in V$ the limit opinion is given by
\begin{equation}
\label{eq:limit-opinion}
\gamma_{\infty}(v)=\sum_{w\in V\setminus C}\gamma_0(w)\alpha_{v,C}(w)+\beta\Theta_{v,C}.
\end{equation}
\end{lemma}
\vvcom{insertar un mono aca con el grafo auxiliar}
%where $\Theta(C)=\sum_{v\in C}\PP_u(\tau=\tau_{v})$.
%Furthermore, we have $\sum_{u\in V\setminus C}\PP_v(\tau_{x_u}=\tau)+\Theta(C)=1$, therefore, 
That is, $\gamma_{\infty}(v)$ is a convex combination of the intrinsic opinions of the nodes in $V\setminus C$, and the opinion $\beta$ of the coordination. 
Observe that for every $v\in V\setminus C$, the first term in the equality above is independent of $\beta$.
We call this term the {\it effective external opinion},
\begin{equation}
\label{eq:limit-0}
\tau_{v,C}(\gamma_0)=\sum_{w\in V\setminus C}\gamma_0(w)\alpha_{v,C}(w).
\end{equation}
For each $C\subseteq V$, the limit opinion is a function of the intrinsic opinions $\gamma_0$ and $\beta$ and it will be useful in what follows to consider the function mapping a pair $(\gamma_0,\beta)\in [0,1]^V\times [0,1]$ onto $\Omega(C,\beta,\gamma_0)=\alpha_C \gamma_0^{\beta}\in \RR^V$.\\

\noindent{\it Conditional Expectations.} 
Given a coordination $(C,\beta)$, by the definition of the hitting probabilities it follows that for each $v\in V\setminus C$, we have that $1-\Theta_{v,C}=\sum_{w\in V\setminus C}\alpha_{v,C}(w)$.
Consider the probability distribution $f_{v,C}$ over the nodes in $V\setminus C$ such that for each $w\in V\setminus C$ we have 
%\begin{equation*}
$f_{v,C}(w)=\frac{\alpha_{v,C}(w)}{1-\Theta_{v,C}}.$
%\end{equation*}
We denote by $\EE_{v,C}$ the expectation operator from probability distributation above.
Observe that $f_{v,C}$ corresponds to the probability distribution induced by $\alpha_C$ conditional on the random walk $(\mathsf{x}_t)_{t\in \ZZ_+}$ starting at $v$ hitting for the first time $C\cup T$ in a vertex of $T$.
Therefore, given a random variable $\mathsf{a}\sim f_{v,C}$, the effective external opinion corresponds to 
\begin{equation*}
\tau_{v,C}(\gamma_0)=(1-\Theta_{v,C})\EE_{v,C}(\gamma_0(\mathsf{a}))
\end{equation*}

\noindent{\it Mixing time.} A quantity that plays a role in our analysis is the time it requires for the random walk distribution to be very close from the stationary one.
Given two probability distributions $\mu$ and $\nu$ over $\mathcal{V}$, the total variation distance between $\mu$ and $\nu$ is given by
\begin{equation*}
\|\mu-\nu\|_{\tv}=\frac{1}{2}\sum_{u\in \mathcal{V}}|\mu(u)-\nu(u)|.
\end{equation*}
For every every $u\in \mathcal{V}$, we denote by $\delta_u$ the probability distribution such that $\delta_u(u)=1$ and zero otherwise. 
Then, the mixing time of the random walk $(\xsf_t)_{t\in \ZZ_+}$ corresponds to the value
\begin{equation*}
\tmix(\mathcal{G})=\min\Big\{t\in \ZZ_+:\max_{u\in \mathcal{V}}\|\psf^t\delta_u-\pi\|_{\tv}\le 1/e\Big\}.
\end{equation*}
That is, the amount of time it requires for the distribution induced by the random walk to be within a distance of at most $1/e$ from the stationary distribution, no matter the initial state.
The constant is arbitrary, since one could replace its value by $\varepsilon$ at cost of a logarithmic factor, $\tmix(\mathcal{G})\log(1/\varepsilon)$.

\section{Coordination Sustainability in the Long-run}

%\noindent{\it Posted Opinions.}
%At every time step $t\in \ZZ_+$, every node $v\in V$ {\it declares} an opinion $\gamma_t(v)$.
%Of particular interest are {\it blind posted opinions}, that is, there exists $\gamma_C\in [0,1]$ such that $\gamma_C(t)=\gamma_C$ for every $t\in \ZZ_+$.
%For every node $u\in V$ and $v\in \nsf(u)$, consider an {\it interaction cost} function $\cost(y(u),y(v))$ that captures the cost incurred by $u$ and $v$, where the vector $y$ represents the {\it state} or {\it configuration} of the graph.
%That is, $\cost(y(u),y(v))$ corresponds to the cost incurred by $u$ and $v$ when they are at opinion states $y(u)$ and $y(v)$ respectively.
%In what follows we assume that the $\cost$ satisfies the following properties,
%\begin{itemize}
%	\item[$(1)$] $\cost(x,x)=0$ for every $x\in \RR$, 
%	\item[$(2)$] $\cost(x,y)=\cost(y,x)$ for every $x,y\in \RR$,
%	\item[$(3)$] $\cost(x,y)\le \cost(x,z)+\cost(z,y)$ for every $x,y,z\in \RR$, and
%	\item[$(4)$] $\cost(x,y)=\cost(1-x,1-y)$ for every $x,y\in [0,1]$.
%\end{itemize}
%\vvcom{maybe the only costs satisfying these four properties are $f(|x-y|)$ with $f$ convex}
%%Every cost function with these two properties is called in what follows a {\it symmetric interaction cost.}
%%\vvcom{deberiamos dar un nombre a esta opinion, "natural" quizas?}
%%That is, it corresponds to the {\it opinion dispersion} faced by node $u$ respect to its neighbors. 
%%Given a graph configuration $y$ and a node $u\in V$, we say that zero is a {\it best response} if  
%%\begin{equation*}
%%\sum_{v\in \nsf(u)}\cost(0,y(v))\le \sum_{v\in \nsf(u)}\cost(\varepsilon,y(v)),
%%\end{equation*} 
%%for every $\varepsilon\in [0,1]$.
%%That is, the dispersion cost of node $u$ at opinion zero is less than the dispersion cost it would incur by setting a different opinion state.
%%and its initial opinion $\gamma_0(u)$. 
%%In what follows we denote by $\degsf(v)=|\nsf(v)|$ the number of neighbors of $v$ in $G$.
%%\vvcom{un poco mas de blabla aca, chequear si es el lenguaje mas apropiado}
%%A large value of $\eta_u$ implies a large cost incurred by $u$.
%%In particular, one can check that for every $t\in \ZZ_+$, $\gamma_v(t)$ corresponds to the value that minimises the opinion variance of $v$ respect to the current opinion vector $\gamma_{t-1}$, that is, 
Given a node $u\in V$, a vector $y\in \RR^V$ and a the initial state $\gamma_0\in \RR^V$, we denote by $\cost_u(y)$ the total interaction cost between $u$ and its neighbors in $\nsf(u)$ at configuration $y$, that is,
\begin{equation*}
\cost_u(y)=\eta_u(\gamma_0(u)-y(u))^2+\sum_{v\in \nsf(u)}(y(v)-y(u))^2.
\end{equation*}
We study the conditions under which it is possible for a coordination to {\it sustain} in the long-run.
%, that is, the dispersion cost of each member in the coordination is less t.
%
Given a triplet $(C,\beta,\gamma_0)$, for each node $u\in C$ decides between joining the coordination, or do not join.
If they join, their opinion state remains unchanged equal to $\beta$.
If they do not join the coordination, they update their opinion according to the weighted opinion dynamics.
%In the following, we say that $\beta$ is a {\it best response} for the coordination $(C,\gamma_0)$ at the configuration $\Omega^C(\gamma_0,\beta)$ if 
We say that $(C,\beta,\gamma_0)$ is {\it sustainable in the long-run} if for every $u\in C$ we have~\footnote{For notational simplicity we denote by $C-u$ the set $C\setminus \{u\}$.} 
for every $u\in C$ we have that
\begin{equation*}
\cost_u(\Omega(C,\beta,\gamma_0))\le \cost_u(\Omega(C-u,\beta,\gamma_0)).
\end{equation*} 
%where $\widetilde_{\infty}=\Omega_{C-u}(\gamma_0,0)$.
That is, joining the coordination $(C,\beta,\gamma_0)$ is a best response for every $u\in C$.\\
%for every $\varepsilon\in [0,1]$.
%\begin{equation*}
%\cost_u(\Omega^C(\gamma_0,\beta))\le \cost_u(\Omega_{C-u}(\gamma_0,\beta)).
%\end{equation*}
%That is, the dispersion cost that a node in $C$ would incur by setting a different opinion state from zero is larger than the cost of being at the zero opinion state.

\noindent{\it Moderate Opinions.} Of particular interest is the long-run sustainability of a coordination with a moderate opinion, that is $\beta\approx 0,5$.
%We say that set $C\subseteq V$ is $\omega$-cohesive if for every $u\in C$ we have that $\degsf_C(u)\ge \omega\cdot \degsf(u)$.
%This notion was introduced by Morris~\cite{morris} to study best response configurations in local interaction frameworks.
In what follows, we say that a pair $(C,\gamma_0)$ is $\varepsilon$-divergent if for every $u\in C$ we have that 
\begin{equation*}
\frac{1}{\degsf(u)}\sum_{v\in \nsf(u)\setminus C}\frac{\cost(\tau_{v,C}(\gamma_0),\tau_{v,C-u}(\gamma_0))}{\cost(0,\tau_{u,C-u}(\gamma_0))}\le 2\varepsilon.
\end{equation*}
\vvcom{el setting de morris es 0 o 1} 
The above notions allows us to find sufficient conditions that guarantee the sustainability of extreme opinions in the long-run.
The following is the first main result of this section.
\begin{theorem}
\label{thm:longrun}
Let $G=(V,E)$ be a connected graph and let $C\subseteq V$ a $(1/2+\varepsilon)$-cohesive set.
Then, for every $\gamma_0\in \RR^V$ the following holds.
\begin{itemize}
	\item[$(a)$] If $(C,\gamma_0)$ is $\varepsilon$-divergent, there exists an interval $\mathcal{I}_{\varepsilon}(C,\gamma_0)\subseteq [0,1]$ with $0\in \mathcal{I}_{\varepsilon}(C,\gamma_0)$ such that $(C,\beta,\gamma_0)$ is sustainable in the long-run for every $\beta \in \mathcal{I}_{\varepsilon}(C,\gamma_0)$.
	\item[$(b)$]  If $(C,\mathsf{e}-\gamma_0)$ is $\varepsilon$-divergent, there exists an interval $\mathcal{I}_{\varepsilon}(C,\gamma_0)\subseteq [0,1]$ with $1\in \mathcal{I}_{\varepsilon}(C,\gamma_0)$ such that $(C,\beta,\gamma_0)$ is sustainable in the long-run for every $\beta \in \mathcal{I}_{\varepsilon}(C,\gamma_0)$.
\end{itemize}
\end{theorem}

%\noindent{\it Extreme Opinions.} Of particular interest is the long-run sustainability of a coordination with an extreme opinion, that is $\beta\in \{0,1\}$.
%We say that set $C\subseteq V$ is $\omega$-cohesive if for every $u\in C$ we have that $\degsf_C(u)\ge \omega\cdot \degsf(u)$.
%This notion was introduced by Morris~\cite{morris} to study best response configurations in local interaction frameworks.
%In what follows, we say that a pair $(C,\gamma_0)$ is $\varepsilon$-divergent if for every $u\in C$ we have that 
%\begin{equation*}
%\frac{1}{\degsf(u)}\sum_{v\in \nsf(u)\setminus C}\frac{\cost(\tau_{v,C}(\gamma_0),\tau_{v,C-u}(\gamma_0))}{\cost(0,\tau_{u,C-u}(\gamma_0))}\le 2\varepsilon.
%\end{equation*}
%\vvcom{el setting de morris es 0 o 1} 
%The above notions allows us to find sufficient conditions that guarantee the sustainability of extreme opinions in the long-run.
%The following is the first main result of this section.
%\begin{theorem}
%\label{thm:longrun}
%Let $G=(V,E)$ be a connected graph and let $C\subseteq V$ a $(1/2+\varepsilon)$-cohesive set.
%Then, for every $\gamma_0\in \RR^V$ the following holds.
%\begin{itemize}
%	\item[$(a)$] If $(C,\gamma_0)$ is $\varepsilon$-divergent, there exists an interval $\mathcal{I}_{\varepsilon}(C,\gamma_0)\subseteq [0,1]$ with $0\in \mathcal{I}_{\varepsilon}(C,\gamma_0)$ such that $(C,\beta,\gamma_0)$ is sustainable in the long-run for every $\beta \in \mathcal{I}_{\varepsilon}(C,\gamma_0)$.
%	\item[$(b)$]  If $(C,\mathsf{e}-\gamma_0)$ is $\varepsilon$-divergent, there exists an interval $\mathcal{I}_{\varepsilon}(C,\gamma_0)\subseteq [0,1]$ with $1\in \mathcal{I}_{\varepsilon}(C,\gamma_0)$ such that $(C,\beta,\gamma_0)$ is sustainable in the long-run for every $\beta \in \mathcal{I}_{\varepsilon}(C,\gamma_0)$.
%\end{itemize}
%\end{theorem}
%\vvcom{describir aca el intervalo}
%Observe that in particular, an extreme opinion in $\{0,1\}$ makes the coordination sustainable in the long-run as long as $(C,\gamma_0)$ satisfies the contraction condition. 
%In what follows we show how to prove the theorem above. 
%We remark that the conditions above depend on both the pair $(C,\gamma_0)$ and the graph $G$.
%Interestingly, the above provides a condition that is robust on the vector of intrinsic opinions and that guarantees the long-run sustainability of an extreme opinion in $\{0,1\}$, since the condition just depends on $C$ and the graph $G$.  
%%More especiically, we say that a pair $C$ is {\it center contractive} if for every $u\in C$ we have that 
%%\vvcom{link to the cohesive definition of Morris (Contagion, 2000)}
%\begin{equation*}
%\frac{1}{2}\cdot \degsf(u)\ge \sum_{v\in \nsf(u)\setminus C}\frac{1-\Theta_{v,C-u}}{1-\Theta_{u,C-u}}.
%\end{equation*}
%%One particular feature of the above structural result is that whenever a pair $(C,\gamma_0)$ is center contractive it will be guaranteed to be zero contractive or one contractive.
%%That is, the coordination will be able to sustain one the extreme opinions.
%Our second theorem of this section states that a center contractive set can sustain an extreme opinion for every vector of intrinsic opinions. 
%That is, the coordination is robust on the set of intrinsic opinions.
%More specifically, we say that a coordination $C$ is {\it robustly sustainable in the long-run} if for every $\gamma_0\in \RR^V$ at least one of the triplets in $\{(C,0,\gamma_0),(C,1,\gamma_0)\}$ is sustainable in the long-run. 
%\begin{theorem}
%\label{thm:alternatives}
%Let $G=(V,E)$ a connected graph and let $C\subseteq V$ be center contractive.
%Then, the coordination $C$ is robustly sustainable in the longrun.
%\end{theorem}
%The above contractive properties are intimately related to the concept of {\it cohesiveness} introduced by Morris~\cite{morris}.
%A set $C$ is {\it $\lambda$-cohesive} if for every $u\in C$ we have that $\degsf_C(u)\ge \lambda\cdot \degsf(u)$.
%Observe that equivalently, a set $C$ is center contractive when for every $u\in C$ we have
%\begin{equation*}
%\degsf_C(u)\ge \left(\frac{1}{2}+\xi_C(u)\right)\degsf(u),
%\end{equation*}
%where $\xi_C(u)$ is equal to 
%\begin{equation*}
%\sum_{v\in \nsf(u)\setminus C}\frac{\Theta_{u,C-u}-\Theta_{v,C-u}}{1-\Theta_{u,C-u}}.
%\end{equation*}
Morris characterizes the contagion threshold of local interaction models in terms of the cohesiveness of {\color{red} blabla}~\cite{morris}.
Recently, Chandrasekhar et al study a model of incomplete information for social learning. 
They show that {\color{red} blabla}.\vvcom{hablar de los clanes de Xandri aca}
In the rest of this section we prove Theorem~\ref{thm:longrun}.
% and Theorem~\ref{thm:alternatives}.
Later, in Section~\ref{sec:random} we study the conditions that guarantee sustainability in the long-run of extreme opinions, in randomly generated graphs.
We come back to all these concepts in that section.

\begin{example}
\vvcom{ejemplo aca del caso que en que $C$ es un cluster, es decir, desconectado del resto, ahi la condicion se cumple}
\end{example}

\vvcom{dar intuicion, algun ejemplo para mostrar que $C$ no puede ser tan chico en general, la idea es ver cuando grande tiene que ser}
\subsection{Sustaining an Extreme Opinion: Proof of Theorem~\ref{thm:longrun}}

We study first the conditions under which a node $u\in C$ faces a lower cost by being in the coordination.
For every $u\in C$, consider the function 
\begin{equation*}
f_u(\beta)=\cost_u(\Omega(C,\beta,\gamma_0))- \cost_u(\Omega(C-u,\beta,\gamma_0))
\end{equation*}
In the following, we say that $(C,\gamma_0)$ is {\it one minded} if for every $u\in C$ we have that $f_u(1)<0$.
Symmetrically, we say that $(C,\gamma_0)$ is {\it zero minded} if for every $u\in C$ we have that $f_u(0)<0$.
The following simple observation allows us to restrict attention on the cost behavior at the extreme opinions.
\begin{proposition}
\label{prop:minded}
Let $C\subseteq V$ be a non-empty subset of nodes and $\gamma_0\in \RR^V$.
Then, the following holds.
\begin{itemize}
	\item[$(a)$] If $C$ is one minded, there exists $\beta^{1}\in (0,1)$ such that for every $f_u(\beta)\le 0$ for every $\beta\in [\beta^1,1]$.
	\item[$(b)$] If $C$ is zero minded, there exists $\beta^{0}\in (0,1)$ such that for every $f_u(\beta)\le 0$ for every $\beta\in [0,\beta^0]$.
\end{itemize}
\end{proposition}

\begin{proof}
If $C$ is one minded, by continuity we have that for every $u\in C$ there exists $\beta_u\in (0,1)$ such that $f_u(\beta)\le 0$ for every $\beta\in [\beta_u,1]$. 
In particular, given $\beta^1=\max_{u\in C}\beta_u$, we have that for every $u\in C$ it holds that $f_u(\beta)<0$ for every $\beta\in [\beta^1,1]$.
The proof follows in the same way when $C$ is zero minded.
\end{proof}

In the following proposition we provide a more explicit expression for the costs evaluated in the longrun opinion vectors.
Furthermore, we also show some opinion symmetry property of the cost function: the cost we face under certain coordination value is the same if every opinion spin is changed. 
%A second key ingredient we use in order to prove the theorem is a monotonicity property from the effective external opinions.
Having that, we are ready to prove Theorem~\ref{thm:longrun}.

\begin{proposition}
\label{prop:properties}
Let $G=(V,E)$ be a connected graph, $C\subseteq V$ and $\gamma_0\in [0,1]^V$.
Then, the following holds.
\begin{itemize}
	\item[$(a)$] When $\beta=0$, for every $v\in V\setminus C$ we have $\Omega_v(C,0,\gamma_0)=\tau_{v,C}(\gamma_0)$.
	\item[$(b)$]  For every $u\in C$, we have that 
%\begin{equation*}
 $\displaystyle \cost_u(\Omega(C,0,\gamma_0))=\sum_{v\in \nsf(u)\setminus C}\cost(\tau_{v,C}(\gamma_0),0).$
%\end{equation*}
	\item[$(c)$]  For every $u\in C$, we have that $\cost_u(\Omega({C-u},\gamma_0,0))$ is equal to
\begin{equation*}
\degsf_C(u)\cdot \cost(\tau_{u,C-u}(\gamma_0),0)+\sum_{v\in \nsf(u)\setminus C}\cost(\tau_{v,C-u}(\gamma_0),\tau_{u,C-u}(\gamma_0)).
\end{equation*}
\end{itemize}
\end{proposition}

\begin{proof}[Proof of Proposition~\ref{prop:properties}]
Part $(a)$ comes directly from equalities Lemma~\ref{lem:limit-thm} and equalities (\ref{eq:limit-opinion}) and (\ref{eq:limit-0}).
For part $(b)$, observe that for every $v\in \nsf(u)\cap C$ we have that 
$\Omega_u(C,0,\gamma_0)=\Omega_v(C,0,\gamma_0)=0$.
%$\gamma_{\infty}(u)=\gamma_{\infty}(v)=0$. 
Therefore,
\begin{equation*}
\cost_u(\Omega(C,0,\gamma_0))=\sum_{v\in \nsf(u)\setminus C}\cost(0,\Omega_v(C,0,\gamma_0))=\sum_{v\in \nsf(u)\setminus C}\cost(0,\tau_{v,C}(\gamma_0)),
\end{equation*}
where the last equality comes from part $(a)$.
Finally, for part $(c)$ 
%observe that for the coordination $(C-u,0)$ we have that $\gamma_{\infty}(v)=\tau_{v,C-u}(\gamma_0)$ for every $v\in V\setminus (C-u)$.
%Therefore, 
we have that $\cost_u(\Omega_{C-u}(\gamma_0,0))$ is equal to 
\begin{align*}
&=\sum_{v\in \nsf(u)\cap C}\cost(\tau_{u,C-u}(\gamma_0),0)+\sum_{v\in \nsf(u)\setminus C}\cost(\tau_{v,C-u}(\gamma_0),\tau_{u,C-u}(\gamma_0))\\
&=\degsf_C(u)\cdot \cost(\tau_{u,C-u}(\gamma_0),0)+\sum_{v\in \nsf(u)\setminus C}\cost(\tau_{v,C-u}(\gamma_0),\tau_{u,C-u}(\gamma_0)).\qedhere
%													&=\degsf_C(u)\cdot \tau^2_{u,C-u}(\gamma_0)+\sum_{v\in \nsf(u)\setminus C}\Big(\tau_{v,C-u}(\gamma_0)-\tau_{u,C-u}(\gamma_0)\Big)^2\\
%													&=\degsf(u)\cdot \tau^2_{u,C-u}(\gamma_0)+\sum_{v\in \nsf(u)\setminus C}\tau^2_{v,C-u}(\gamma_0)-2\cdot \tau_{u,C-u}(\gamma_0)\sum_{v\in \nsf(u)\setminus C}\tau_{v,C-u}(\gamma_0).\qedhere
\end{align*}
\end{proof}

\begin{proposition}
\label{prop:spin}
Let $G=(V,E)$ be a connected graph, $C\subseteq V$ and $\gamma_0\in [0,1]^V$.
Then, for every $u\in C$ we have that 
\begin{equation*}
\cost_u(\Omega(C,0,\gamma_0))=\cost_u(\Omega(C,1,\mathsf{e}-\gamma_0)).
\end{equation*}
\end{proposition}

\begin{proof}
Recall that for every $u\in V$, we have that $\Omega_u(C,1,\mathsf{e}-\gamma_0)=\tau_{v,C}(\mathsf{e}-\gamma_0)+\Theta_{v,C}$ when the coordination is given by $(C,1,\mathsf{e}-\gamma_0)$.
Furthermore, we have that 
\begin{align*}
\tau_{v,C}(\mathsf{e}-\gamma_0)+\Theta_{v,C}&=\sum_{w\in V\setminus C}(1-\gamma_0(w))\alpha_C(v,w) +\Theta_{v,C}\\
&=1-\Theta_{v,C}-\tau_{v,C}(\gamma_0)+\Theta_{v,C}=1-\tau_{v,C}(\gamma_0).
\end{align*}
Therefore, we have $\Omega_u(C,1,\mathsf{e}-\gamma_0)=1-\Omega_u(C,0,\gamma_0)$.
In particular, for every $u,v\in V$ we have that 
\begin{align*}
\cost(\Omega_u(C,1,\mathsf{e}-\gamma_0),\Omega_v(C,1,\mathsf{e}-\gamma_0))&=\cost(1-\Omega_u(C,0,\gamma_0),1-\Omega_v(C,0,\gamma_0))\\
&=\cost(\Omega_u(C,0,\gamma_0),\Omega_v(C,0,\gamma_0)),
\end{align*}
where the last equality comes from property $(4)$ of the $\cost$ function.
Therefore, we conclude that 
\begin{align*}
\cost_u(\Omega(C,1,\mathsf{e}-\gamma_0,))&=\sum_{v\in \nsf(u)}\cost(\Omega_u(C,1,\mathsf{e}-\gamma_0),\Omega_v(C,1,\mathsf{e}-\gamma_0))\\
&=\sum_{v\in \nsf(u)}\cost(\Omega_u(C,0,\gamma_0),\Omega_v(C,0,\gamma_0))=\cost_u(\Omega(C,0,\gamma_0)).\qedhere
\end{align*}
%\begin{equation*}
%\cost_u(\Omega^C(\mathsf{e}-\gamma_0,1-\beta))=\sum_{v\in \nsf(u)\cap C}(-(\tau_{}))^2+\sum_{v\in \nsf(u)\setminus C}(\gamma_{\infty}(u)-\gamma_{\infty}(v))^2
%\end{equation*}
\end{proof}
\begin{comment}
As mentioned before, the effective external opinions satisfy a particular monotonicity. 
In words, the effective external opinion of a node out of a coordination is at least the effective external opinion of a node in the coordination.

\begin{lemma}
\label{lem:mono-external}
Let $G=(V,E)$ be a connected graph, $C\subseteq V$ and $\gamma_0\in [0,1]^V$.
Then, for every $u\in C$ and for every $v\in \nsf(u)\setminus C$ we have that $\tau_{v,C}(\gamma_0)\le \tau_{v,C-u}(\gamma_0)$.
\end{lemma}

\begin{proof}
\vvcom{chequear esto}
From the definition of the effective external opinions we have that
\begin{align*}
\tau_{v,C-u}(\gamma_0)&=\sum_{w\in V\setminus (C-u)}\gamma_0(w)\alpha_{v,C-u}(w)= \sum_{w\in V\setminus C}\gamma_0(w)\alpha_{v,C-u}(w)+\gamma_0(u)\alpha_{v,C-u}(u).\end{align*}
Recall that for every $v\in C$, we have that $\alpha_{v,C-u}(w)$ corresponds to the hitting probability given by $\PP_v(\tau_{x_w}=\tau)$ of the random walk $(\xsf_t)_{t\in \ZZ_+}$ starting at $v$ over the auxiliary network $\calG_C$ defined in Section~\ref{sec:auxiliar}.
\begin{claim}
For every $w\in V\setminus C$ we have $\alpha_{v,C-u}(w)\ge \alpha_{u,C-u}(w)\cdot \alpha_{v,C}(u)$.
\end{claim}
\noindent The claim together with the above inequality implies that 
\begin{align*}
\tau_{v,C-u}(\gamma_0)&=\sum_{w\in V\setminus C}\gamma_0(w)\alpha_{v,C-u}(w)+\gamma_0(u)\alpha_{v,C-u}(u)\\
				&=\sum_{w\in V\setminus C}\gamma_0(w)\alpha_{u,C-u}(w)\cdot\alpha_{v,C}(u)+\gamma_0(u)\alpha_{v,C-u}(u)\\
				&=\tau_{u,C-u}(\gamma_0)\alpha_{v,C}(u)+\gamma_0(u)\alpha_{v,C-u}(u)\ge \tau_{u,C-u}(\gamma_0)\alpha_{v,C}(u).
\end{align*}
In what follows we show that the claim above holds.
Fix a node $w\in V\setminus C$. 
By conditioning we have that 
\begin{align*}
\alpha_{v,C-u}(w)&=\PP_v(\tau_{x_w}=\tau_{C-u})\\
			&\ge \PP_{v}(\tau_{x_w}=\tau_{C-u}|\tau_u=\tau_C)\cdot\PP_v(\tau_{u}=\tau_{C})\ge \alpha_{u,C-u}(w)\cdot\alpha_{v,C}(u).\qedhere
\end{align*}
\end{proof}
\end{comment}
\begin{proof}[Proof of Theorem~\ref{thm:longrun}]
\vvcom{arreglar esto}
In what follows, let $(C,\gamma_0)$ be a pair $\varepsilon$-divergent.
We show next that $(C,\gamma_0)$ is zero minded.
%Suppose first that $(C,\gamma_0)$ satisfies the zero contraction property. 
Observe that since $\cost$ satisfies symmetry and the triangular inequality properties $(2)$ and $(3)$, we have that for every $u\in C$ and for every $v\in \nsf(u)$,
\begin{equation*}
\cost(\tau_{v,C}(\gamma_0),0)\le \cost(\tau_{v,C}(\gamma_0),\tau_{v,C-u}(\gamma_0))+\cost(\tau_{v,C-u}(\gamma_0),\tau_{u,C-u}(\gamma_0))+\cost(\tau_{u,C-u}(\gamma_0),0).
\end{equation*}
The above inequality together with Proposition~\ref{prop:properties}, implies that for every $u\in C$ we have that
\begin{align*}
\frac{f_u(0)}{\cost(\tau_{u,C-u}(\gamma_0),0)}&\le \sum_{v\in \nsf(u)\setminus C}\frac{ \cost(\tau_{v,C}(\gamma_0),\tau_{v,C-u}(\gamma_0))}{\cost(\tau_{u,C-u}(\gamma_0),0)}-\degsf_C(u)+\degsf(u)-\degsf_C(u)\\
&\le 2\varepsilon\cdot \degsf(u)+\degsf(u)-2\degsf_C(u)=(2\varepsilon+1)\degsf(u)-2\degsf_C(u),
\end{align*}
where the second inequality comes from the fact that $(C,\gamma_0)$ is $\varepsilon$-divergent.
Finally, since $C$ is $(1/2+\varepsilon)$-cohesive we conclude that 
\begin{equation*}
\frac{f_u(0)}{2\cdot \cost(\tau_{u,C-u}(\gamma_0),0)}\le \left(\frac{1}{2}+\varepsilon\right)\degsf(u)-\degsf_C(u)\le 0.
\end{equation*}
It follows that $f_u(0)\le 0$ for every $u\in C$, and therefore $C$ is zero minded. 
By Proposition~\ref{prop:minded} there exists an interval $[0,\beta^0]$ such that $f_u(\beta)\le 0$ for every $u\in C$ and for every $\beta\in [0,\beta^0]$. 
In this case the theorem follows by taking $\mathcal{I}_{\varepsilon}(C,\gamma_0)=[0,\beta^0]$.
Now suppose that $(C,\mathsf{e}-\gamma_0)$ is $\varepsilon$-divergent. 
Observe that by Proposition~\ref{prop:spin} we have that 
\begin{align*}
f_u(1)&=\cost_u(\Omega(C,1,\gamma_0))-\cost_u(\Omega(C-u,1,\gamma_0))\\
&=\cost_u(\Omega(C,0,\mathsf{e}-\gamma_0))-\cost_u(\Omega(C-u,0,\mathsf{e}-\gamma_0)).
\end{align*}
On the other hand, since $(C,\mathsf{e}-\gamma_0)$ is $\varepsilon$-divergent, by the previous case we conclude that $f_u(1)\le 0$ and therefore $C$ is one minded. 
By Proposition~\ref{prop:minded} there exists an interval $[\beta^1,1]$ such that $f_u(\beta)\le 0$ for every $u\in C$ and for every $\beta\in [\beta^1,1]$. 
The theorem follows by taking $\mathcal{I}_{\varepsilon}(C,\gamma_0)=[\beta^1,1]$.
\end{proof}

\begin{comment}
\subsection{Robust Sustainability: Proof of Theorem~\ref{thm:alternatives}}
%
\vvcom{arreglar esto}
In order to prove the theorem, in the following we first show that if $C\subseteq V$ is center contractive then it should be zero contractive or one contractive.
%That is, the coordination can sustain only one of the extreme opinions in the longrun.
%This theorem can be seen as an alternatives result, since only one of two possibilities can hold.
%Suppose that $(C,\gamma_0)$ is both zero contractive and one contractive. 
Given a set $C\subseteq V$ and $u\in C$, consider the linear function 
\begin{equation*}
\Phi_{u,C}(y)=\sum_{v\in \nsf(u)\setminus C}\tau_{v,C-u}(y)-\frac{1}{2}\degsf(u)\tau_{u,C-u}(y)
\end{equation*}
In the following lemma we characterize the three notions of being contractive by using the linear function above.
Having that we are ready to prove the theorem.
%Consider the set of vectors given by $\calF(C)=\left\{y\in \RR^V:\Phi_u(y,C)\le 0\text{ for every }u\in C\right\}$.
\begin{lemma}
\label{lem:convexity}
Let $G=(V,E)$ be a connected graph, $C\subseteq V$ and $\gamma_0\in [0,1]^V$.
Then the following holds.
\begin{itemize}
	\item[$(a)$] The pair $(C,\gamma_0)$ is zero contractive if and only if $\Phi_{u,C}(\gamma_0)\le 0$ for every $u\in C$.
	\item[$(b)$] The pair $(C,\gamma_0)$ is one contractive if and only if $\Phi_{u,C}(\mathsf{e}-\gamma_0)\le 0$ for every $u\in C$.
	\item[$(c)$] The set $C$ is center contractive if and only if $\Phi_{u,C}(\mathsf{e})\le 0$ for every $u\in C$.
\end{itemize}
\end{lemma}

\begin{proof}
The statements $(a)$ and $(b)$ follows directly from the definition of zero and one contractive pairs.
For part $(c)$, observe that for every $u\in C$ and for every $v\in \nsf(u)\setminus C$, we have that 
\begin{align*}
\tau_{v,C-u}(\mathsf{e})&=\sum_{w\in V\setminus (C-u)}\alpha_{v,C-u}(w)=1-\Theta_{v,C-u},\\
\tau_{u,C-u}(\mathsf{e})&=\sum_{w\in V\setminus (C-u)}\alpha_{u,C-u}(w)=1-\Theta_{u,C-u}.
\end{align*}
Therefore, $\Phi_{u,C}(\mathsf{e})\le 0$ for every $u\in C$ if and only if 
\begin{equation*}
\sum_{v\in \nsf(u)\setminus C}\Big(1-\Theta_{v,C-u}\Big)-\frac{1}{2}\degsf(u)\Big(1-\Theta_{u,C-u}\Big)\le 0
\end{equation*}
for every $u\in C$, which corresponds to $C$ being center contractive.
\end{proof}

\begin{proof}[Proof of Theorem~\ref{thm:alternatives}]
Suppose that the set $C$ is center contractive but there exists $\gamma_0\in \RR^V$ such that $(C,\gamma_0)$ is neither zero contractive nor one contractive.
By Lemma~\ref{lem:convexity}, we have that $\Phi_{u,C}(\gamma_0)> 0$ and $\Phi_{u,C}(\mathsf{e}-\gamma_0)> 0$ for every $u\in C$.
By linearity, for every $u\in C$ it follows that 
\begin{equation*}
0<\Phi_{u,C}(\gamma_0)+\Phi_{u,C}(\mathsf{e}-\gamma_0)=\Phi_{u,C}(\mathsf{e}),
\end{equation*}
which by Lemma~\ref{lem:convexity} contradicts the fact that $C$ is center contractive.
Namely, we showed that if $C$ is center contractive then for every $\gamma_0\in \RR^V$ the pair $(C,\gamma_0)$ is zero contractive or one contractive. 
That is, $C$ is robustly sustainable in the long-run.
\end{proof}

\end{comment}
\begin{comment}
Given a triplet $(C,\beta,\gamma_0)$, observe that for every $u\in C$ that
\begin{equation*}
f_u''(\beta)=\frac{\eta_u}{2}\Big(1-\Theta(C-u)^2\Big)+\frac{1}{2}\Big(\deg(u)-\deg_C(u)\Big)(1-\Theta(C))^2-\frac{1}{2}\deg_C(u)(1-\Theta(C-u))^2.
\end{equation*}

\begin{equation}
f_{\beta}(u) = \frac{\eta_u}{2}(\gamma_0-\beta)^2 + \sum_{v\in \nsf(u)\setminus C}
((1-\Theta(C))\beta+g_v)^2-\frac{\eta_u}{2} (\gamma_0-\tilde{g}_u-\beta \Theta(C-u) )^2\\
 - \frac{1}{2} \deg_C(u) ((1-\Theta(C-u))\beta-\tilde{g}_u)^2
\end{equation}
where,
\begin{align*}
	g_v &=\sum_{w \in V \setminus C } \gamma_0(w) \alpha_C(v,w)\\
	\tilde{g}_u &=\sum_{w \in V \setminus (C-u) } \gamma_0(w) \alpha_{C-u}(u,w)
\end{align*}
Note that $f_{\beta}(u)$ is a quadratic function. When $\eta_u=0$ this function is concave if $\frac{\deg(u)}{1+\rho^2(C,u)}\leq \deg_C(u)$ where $\rho(C,u)=\frac{1-\Theta(C-u)}{1-\Theta(C)}\geq 1$. When $\eta_u>0$ the concavity is given by ,
\begin{align*}
\eta_u \left( \frac{1-\Theta^2(C-u)}{(1-\Theta(C))^2} \right) \frac{1}{1+\rho^2(C,u)}+\frac{\deg(u)}{1+\rho^2(C,u)}<\deg_C(u)
\end{align*}

In particular, if $\forall u \in C$ 

\begin{align}
\frac{\eta_u+\deg(u)}{2} \leq \deg_C(u)
\end{align}
we have that $f_{\beta}(u)$ is a concave function.º

\begin{proposition}
Let $G=(V,E)$ a connected graph, a vector of intrinsic opinions $\gamma_0\in [0,1]^V$ and $C\subseteq V$.
Then, for every $u\in C$ there exists an interval $\mathcal{I}_{u,C}(\gamma_0)$, such that $f_u(\beta)\le 0$ if and only if $\beta\in \mathcal{I}_{u,C}(\gamma_0)$.
\end{proposition}
\end{comment}

\begin{comment}
\section{Sustainability in the Long-run: Random Intrinsic Opinions}

In what follows, we study under what conditions it is possible for a coordination to sustain in the long-run, that is, by considering the cost under the limiting opinion.
The intrinsic opinions are drawn independently and uniformly distributed from $[0,1]$.
We say that $(C,\beta,\gamma_0)$ is {\it sustainable in the long-run} if for every $u\in C$ we have~\footnote{For notational simplicity we denote by $C-u$ the set $C\setminus \{u\}$.} 
\begin{equation*}
\EE_{\gamma_0}(\cost_u(\Omega^C(\gamma_0,\beta)))\le \EE_{\gamma_0}(\cost_u(\Omega_{C-u}(\gamma_0,\beta))).
\end{equation*}
That is, every node in $C$ faces a lower cost by being in the coordination than being out.
In what follows, we say that a pair $(C,\gamma_0)$ satisfies the {\it zero contraction condition} if for every $u\in C$, we have that 
%\vvcom{Aca va la condicion que garantiza que $C$ es one minded o zero minded (ver mas abajo la def de esto)}
%\vvcom{estas son las condiciones para $\eta_u=0$, cuando son distintos de cero queda un poco mas fea pero no tanto. Pero bueno, es lo que es no mas}
\begin{equation*}
%\EE^2_{u,C-u}(\gamma_0(\mathsf{a}_u))\ge \frac{1}{\deg_C(u)}\sum_{v\in V\setminus C}\EE_{v,C}^2(\gamma_0(\mathsf{a}_v))\left(\frac{1-\Theta_v(C)}{1-\Theta_u(C-u)}\right)^2
%\deg_C(u) \ge \sum_{v\in V\setminus C}\left(\frac{\tau_{v,C}(\gamma_0)}{\tau_{u,C-u}(\gamma_0)}\right)^2+\eta_u\left(\frac{2\gamma_0(u)}{\tau_{u,C-u}(\gamma_0)}-1\right).
%\degsf_C(u) + \sum_{v\in V\setminus C}\left(\frac{1-\Theta_{v,C-u}}{1-\Theta_{u,C-u}}-1\right)^2\ge \sum_{v\in V\setminus C}\left(\frac{1-\Theta_{v,C}}{1-\Theta_{u,C-u}}\right)^2+\eta_u\frac{1+\Theta_{u,C-u}}{1-\Theta_{u,C-u}}.
\degsf_C(u)\ge \max\Big\{\|\alpha_{u,C-u}\|_1^2\Phi_1(u,C),\|\alpha_{u,C-u}\|_2^2\Phi_2(u,C)\Big\},
\end{equation*}
where the $\Phi_1(u,C)$ and $\Phi_2(u,C)$ are the graph parameters given by 
\begin{align*}
%\Phi_1(u,C)&=\sum_{v\in V\setminus C}\left(\left(\frac{1-\Theta_{v,C}}{1-\Theta_{u,C-u}}\right)^2-\left(\frac{1-\Theta_{v,C-u}}{1-\Theta_{u,C-u}}-1\right)^2\right)+\eta_u\frac{1+\Theta_{u,C-u}}{1-\Theta_{u,C-u}},\\
\Phi_1(u,C)&=\sum_{v\in \nsf(u)\setminus C}\|\alpha_{v,C}\|_1^2-\sum_{v\in \nsf(u)\setminus C}\|\alpha_{v,C-u}-\alpha_{u,C-u}\|_1^2.\\
\Phi_2(u,C)&=\sum_{v\in \nsf(u)\setminus C}\|\alpha_{v,C}\|_2^2-\sum_{v\in \nsf(u)\setminus C}\|\alpha_{v,C-u}-\alpha_{u,C-u}\|_2^2.
\end{align*}
%where $\mathsf{a}_u\sim f_{u,C-u}$ and $\mathsf{a}_{v}\sim f_{v,C}$ for every $v\in V\setminus C$.
Similarly, we say that a pair $(C,\gamma_0)$ satisfies the {\it one opinion contraction condition} if for every $u\in C$, we have that 
%\vvcom{Aca va la condicion que garantiza que $C$ es one minded o zero minded (ver mas abajo la def de esto)}
\begin{equation*}
%\EE^2_{u,C-u}(\gamma_0(\mathsf{a}_u))\ge \frac{1}{\deg_C(u)}\sum_{v\in V\setminus C}\EE_{v,C}^2(\gamma_0(\mathsf{a}_v))\left(\frac{1-\Theta_v(C)}{1-\Theta_u(C-u)}\right)^2
%\deg_C(u) \ge \sum_{v\in V\setminus C}\left(\frac{\tau_{v,C}(\mathsf{e}-\gamma_0)}{\tau_{u,C-u}(\mathsf{e}-\gamma_0)}\right)^2+\eta_u\left(\frac{2(1-\gamma_0(u))}{\tau_{u,C-u}(\mathsf{e}-\gamma_0)}-1\right),
\degsf_C(u) + \sum_{v\in V\setminus C}\left(\frac{\tau_{v,C-u}(\mathsf{e}-\gamma_0)}{\tau_{u,C-u}(\mathsf{e}-\gamma_0)}-1\right)^2\ge \sum_{v\in V\setminus C}\left(\frac{\tau_{v,C}(\mathsf{e}-\gamma_0)}{\tau_{u,C-u}(\mathsf{e}-\gamma_0)}\right)^2.
\end{equation*}
\vvcom{me da la sensacion de que un conjunto no puede zer zero contractive y one contractive a la vez}
where $\mathsf{e}$ is the all ones vector in $\RR^V$.
In the sequel, we say that $(C,\gamma_0)$ satisfies the contraction condition if it satisfies at least one of the above conditions.
The following is the main theorem of this section.
\begin{theorem}
\label{thm:longrun-random}
Let $G=(V,E)$ a connected graph and $(C,\gamma_0)$ satisfying the contraction condition.
Then, there exists a non-empty interval $\mathcal{I}(C,\gamma_0)\subseteq [0,1]$ with $\mathcal{I}(C,\gamma_0)\cap \{0,1\}\ne \emptyset$ such that for every $\beta\in \mathcal{I}(C,\gamma_0)$ we have that $(C,\beta,\gamma_0)$ is sustainable in the long-run.  
\end{theorem}
%\vvcom{describir aca el intervalo}
Observe that in particular, an extreme opinion in $\{0,1\}$ makes the coordination sustainable in the long-run as long as $(C,\gamma_0)$ satisfies the contraction condition. 
In what follows we show how to prove the theorem above. 
\vvcom{aqui discutir un poco el teorema anterior}
\subsection{Extreme Opinion Minded Sets: Proof of Theorem~\ref{thm:longrun-random}}

We study first the conditions under which a node $u\in C$ faces a lower cost by being in the coordination.
For every $u\in C$, consider the function 
\begin{equation*}
f_u(\beta)=\EE_{\gamma_0}(\cost_u(\Omega^C(\gamma_0,\beta)))- \EE_{\gamma_0}(\cost_u(\Omega_{C-u}(\gamma_0,\beta))).
\end{equation*}
In the following, we say that $C$ is {\it one minded} if for every $u\in C$ we have that $f_u(1)<0$.
Symmetrically, we say that $C$ is {\it zero minded} if for every $u\in C$ we have that $f_u(0)<0$.

\begin{proposition}
\label{prop:minded}
Let $C\subseteq V$ be a non-empty subset of nodes.
Then, the following holds.
\begin{itemize}
	\item[$(a)$] If $C$ is one minded, there exists $\beta^{1}\in (0,1)$ such that for every $f_u(\beta)\le 0$ for every $\beta\in [\beta^1,1]$.
	\item[$(b)$] If $C$ is zero minded, there exists $\beta^{0}\in (0,1)$ such that for every $f_u(\beta)\le 0$ for every $\beta\in [0,\beta^0]$.
\end{itemize}
\end{proposition}

\begin{proof}
If $C$ is one minded, by continuity we have that for every $u\in C$ there exists $\beta_u\in (0,1)$ such that $f_u(\beta)\le 0$ for every $\beta\in [\beta_u,1]$. 
In particular, given $\beta^1=\max_{u\in C}\beta_u$, we have that for every $u\in C$ it holds that $f_u(\beta)<0$ for every $\beta\in [\beta^1,1]$.
The proof follows in the same way when $C$ is zero minded.
\end{proof}

In the following propositions we provide a more explicit expression for the costs evaluated in the longrun opinion vectors. 
Having that, we are ready to prove Theorem~\ref{thm:longrun}.
One [property that will be useful in what comes next is the following. 
Consider a random vector $\xi\in \RR^m$ where every cordinate is independently and uniformly distributed over $[0,1]$, and consider a vector a vector $x\in \RR^m$.
Then, we have that
\begin{equation*}
\EE_{\xi}(\langle \xi,x\rangle)^2=\frac{1}{12}\|x\|_2^2+\|x\|_1^2.
\end{equation*}

\begin{lemma}
\label{lem:expected-limit}
Let $G=(V,E)$ a connected graph and $\gamma_0$ drawn independently and uniformly from $[0,1]$. 
Then, the following holds.
\begin{itemize}
	\item[$(a)$] For every $v\in V\setminus C$ we have that $\EE_{\gamma_0}(\tau_{v,C}(\gamma_0))=\EE_{\gamma_0}(\tau_{v,C}(\mathsf{e}-\gamma_0))=\frac{1}{2}(1-\Theta_v(C))$.
	\item[$(b)$] For every $v\in V\setminus C$ we have that $\EE_{\gamma_0}(\tau^2_{v,C}(\gamma_0))=\frac{1}{12}\|\alpha_{v,C}\|_2^2+\EE^2_{\gamma_0}(\tau_{v,C}(\gamma_0))$.
	\item[$(c)$] For every $u\in C$ and every $v\in V\setminus (C-u)$ we have that 
	\begin{equation*}
	\EE_{\gamma_0}\Big(\tau_{v,C-u}(\gamma_0)-\tau_{u,C-u}(\gamma_0)\Big)^2=\frac{1}{12}\|\alpha_{v,C-u}-\alpha_{u,C-u}\|_2^2+\frac{1}{2}(\Theta_{u,C-u}-\Theta_{v,C-u})^2.
	\end{equation*}
%	\item[$(d)$] For every $u\in C$ we have that $\EE_{\gamma_0}(\gamma_0(u)\tau_{u,C-u}(\gamma_0))=\frac{1}{12}\alpha_{u,C-u}(u)+\frac{1}{4}(1-\Theta_{u,C-u})$.
	\item[$(d)$] For every $v\in V\setminus C$ we have that $\EE_{\gamma_0}(\gamma_{\infty}(v))=\frac{1}{2}+\left(\beta-\frac{1}{2}\right)\Theta_v(C)$.
\end{itemize}
\end{lemma}

\begin{proof}[Proof of Lemma~\ref{lem:expected-limit}]

\end{proof}

\begin{proposition}
\label{prop:properties}
Let $G=(V,E)$ be a connected graph, $C\subseteq V$ and $\gamma_0\in [0,1]^V$.
Then, the following holds.
\begin{itemize}
	\item[$(a)$] When $\beta=0$, for every $v\in V\setminus C$ we have $\gamma_{\infty}(v)=\tau_{v,C}(\gamma_0)$.
	\item[$(b)$]  For every $u\in C$, we have that 
\begin{equation*}
\EE_{\gamma_0}(\cost_u(\Omega^C(\gamma_0,0)))=\sum_{v\in \nsf(u)\setminus C}\EE_{\gamma_0}(\tau^2_{v,C}(\gamma_0))
\end{equation*}
	\item[$(c)$]  For every $u\in C$, we have that $\cost_u(\Omega_{C-u}(\gamma_0,0))$ is equal to
\begin{equation*}
\degsf_C(u)\cdot \EE_{\gamma_0}(\tau^2_{u,C-u}(\gamma_0))+\sum_{v\in \nsf(u)\setminus C}\EE_{\gamma_0}\Big(\tau_{v,C-u}(\gamma_0)-\tau_{u,C-u}(\gamma_0)\Big)^2.
%\eta_u\left(\frac{\gamma_0(u)}{\tau_{u,C-u}(\gamma_0)}-1\right)^2+\degsf_C(u)+\sum_{v\in V\setminus C}\left(\frac{\tau_{v,C-u}(\gamma_0)}{\tau_{u,C-u}(\gamma_0)}-1\right)^2.
\end{equation*}
\end{itemize}
\end{proposition}

\begin{proof}[Proof of Proposition~\ref{prop:properties}]
\vvcom{insertar demo de cada punto}
\end{proof}


\begin{proof}[Proof of Theorem~\ref{thm:longrun}]
In what follows, we show that if a pair $(C,\gamma_0)$ satisfies the contraction property, then it is either zero minded or one minded. 
Suppose first that $(C,\gamma_0)$ satisfies the zero contraction property. 
Thanks to Proposition~\ref{prop:properties}, for every $u\in C$ we have that $\tau^{-2}_{u,C-u}(\gamma_0)f_u(0)$ is equal to
\begin{equation*}
\sum_{v\in V\setminus C}\left(\frac{\tau_{v,C}(\gamma_0)}{\tau_{u,C-u}(\gamma_0)}\right)^2-\degsf_C(u) - \sum_{v\in V\setminus C}\left(\frac{\tau_{v,C-u}(\gamma_0)}{\tau_{u,C-u}(\gamma_0)}-1\right)^2\le 0,
\end{equation*}
where the last inequality comes from the fact that $(C,\gamma_0)$ satisfies the zero contraction property.
It follows that $f_u(0)\le 0$ for every $u\in C$, and therefore $C$ is zero minded. 
By Proposition~\ref{prop:minded} there exists an interval $[0,\beta^0]$ such that $f_u(\beta)\le 0$ for every $u\in C$ and for every $\beta\in [0,\beta^0]$. 
In this case the theorem follows by taking $\mathcal{I}(C,\gamma_0)=[0,\beta^0]$.
\end{proof}
\end{comment}

%\begin{proof}
%
%\end{proof}
%
%\begin{proof}[Proof of Theorem~\ref{thm:longrun}]
%{\color{red}tomamos la interseccion sobre todos los intervalos, $\mathcal{I}_{\gamma_0,C}=\cap_{u\in C}\mathcal{I}_u$. La condicion sobre $C$ es tal que esta interseccion es no vacia.} 
%\end{proof}
\begin{comment}
Given a triplet $(C,\beta,\gamma_0)$, observe that for every $u\in C$ that
\begin{equation*}
f_u''(\beta)=\frac{\eta_u}{2}\Big(1-\Theta(C-u)^2\Big)+\frac{1}{2}\Big(\deg(u)-\deg_C(u)\Big)(1-\Theta(C))^2-\frac{1}{2}\deg_C(u)(1-\Theta(C-u))^2.
\end{equation*}

\begin{equation}
f_{\beta}(u) = \frac{\eta_u}{2}(\gamma_0-\beta)^2 + \sum_{v\in \nsf(u)\setminus C}
((1-\Theta(C))\beta+g_v)^2-\frac{\eta_u}{2} (\gamma_0-\tilde{g}_u-\beta \Theta(C-u) )^2\\
 - \frac{1}{2} \deg_C(u) ((1-\Theta(C-u))\beta-\tilde{g}_u)^2
\end{equation}
where,
\begin{align*}
	g_v &=\sum_{w \in V \setminus C } \gamma_0(w) \alpha_C(v,w)\\
	\tilde{g}_u &=\sum_{w \in V \setminus (C-u) } \gamma_0(w) \alpha_{C-u}(u,w)
\end{align*}
Note that $f_{\beta}(u)$ is a quadratic function. When $\eta_u=0$ this function is concave if $\frac{\deg(u)}{1+\rho^2(C,u)}\leq \deg_C(u)$ where $\rho(C,u)=\frac{1-\Theta(C-u)}{1-\Theta(C)}\geq 1$. When $\eta_u>0$ the concavity is given by ,
\begin{align*}
\eta_u \left( \frac{1-\Theta^2(C-u)}{(1-\Theta(C))^2} \right) \frac{1}{1+\rho^2(C,u)}+\frac{\deg(u)}{1+\rho^2(C,u)}<\deg_C(u)
\end{align*}

In particular, if $\forall u \in C$ 

\begin{align}
\frac{\eta_u+\deg(u)}{2} \leq \deg_C(u)
\end{align}
we have that $f_{\beta}(u)$ is a concave function.º

\begin{proposition}
Let $G=(V,E)$ a connected graph, a vector of intrinsic opinions $\gamma_0\in [0,1]^V$ and $C\subseteq V$.
Then, for every $u\in C$ there exists an interval $\mathcal{I}_{u,C}(\gamma_0)$, such that $f_u(\beta)\le 0$ if and only if $\beta\in \mathcal{I}_{u,C}(\gamma_0)$.
\end{proposition}
\end{comment}



\section{Random Networks and Robust Sustainability}
\label{sec:random}

In the following we analyze long-run sustainability coordination over randomly generated networks.
The first random graph family we consider is given by a mixture between the classic Erdos-Renyi model and the Geometric model. 
This family has been studied in the context of social learning on networks {\color{red} blabla aca}.
We show that under certain regime of the random graph model, with high probability there exists a set that is center contractive.
%In particular, Theorem~\ref{thm:alternatives} guarantees the robust sustainability on the longrun for the extreme opinions on such coordination.

\subsection{Cohesiveness and Center Contractiveness}

%\noindent{\it Cohesive Graphs.} 
Recall that a set $C\subseteq V$ is said to be $\omega$-cohesive if for every $u\in C$ we have that $\degsf_C(u)\ge \omega\cdot \degsf(u)$.
This notion was introduced by Morris~\cite{morris} in a more general interaction framework in order to study contagion processes.
In what follows we say that a node $u\in V$ is {\it $(\omega,k)$-cohesive} if there exists a $\omega$-cohesive set $C\subseteq V$ of size at most $k$ such that $u\in C$.
We say that a graph $G$ is $(\mu,\omega,k)$-cohesive if the fraction of nodes that are $(\omega,k)$-cohesive is at least $\mu$.
%That is, at least a $\mu$ fraction of the nodes belong to some subset of nodes of size at most $k$ that is $\lambda$-cohesive.
Recently, Chandrasekhar et al~\cite{} study the existence of $1/2$-cohesive sets in a family of randomly generated graphs. 
They call such a set a {\it clan}. 
In the next section we go back to this point.

\begin{example}
\vvcom{ejemplo aca de un $(1/2-eps,1/eps)$-cohesive graph, tomar el del libro-paper de Young}
\end{example}
We remark that this notion is closely related to the {\it close-knit graphs} introduced by Young~\cite{young:innovation} in order to study the long-run behavior  of dynamics associated to local interactions systems. 
In particular, one can show that $(\omega,k)$-close-knit graphs are $(1,\omega,k)$-cohesive graphs.
The following lemma relates the cohesiveness property and the fact of being center contractive.
It will be useful late in our analysis.
Recall that given $C\subseteq V$, the value $\xi_C(u)$ is given by $\sum_{v\in \nsf(u)\setminus C}\frac{\Theta_{u,C-u}-\Theta_{v,C-u}}{1-\Theta_{u,C-u}}.
$
We denote by $\xi_C=\max_{u\in C}\xi_C(u)$ the maximum of this quantity over every node in $C$, and we call it the {\it shifting parameter} of $C$.
\begin{proposition}
\label{prop:coh-cont}
Let $G=(V,E)$ be a connected graph.
If $C\subseteq V$ is $(1/2+\xi_C)$-cohesive then it is center contractive as well.
\end{proposition}

\begin{proof}
If $C$ is $(1/2+\xi_C)$-cohesive we have that $\degsf_C(u)\ge (1/2+\xi_C)\degsf(u)$ for every $u\in C$.
In particular, since $\xi_C\ge \xi_C(u)$ for all $u\in C$, it follows that $\degsf_C(u)\ge (1/2+\xi_C(u))\degsf(u)$ and therefore $C$ is center contractive.
\end{proof}

In the following, we say that graph is {\it $(\omega,k)$-centered contractive} if at least an $\omega\in [0,1]$ fraction of the nodes belong to a center contractive set of at most $k$.
In particular, if there exists a center contractive set of size $k$ the graph is $(k/n,k)$-center contractive.
On the other hand, every connected graph of $n$ nodes is $(1,n)$-center contractive since the set of nodes is clearly center contractive.
The goal is to study when we can guarantee that a large enough fraction of the nodes belong to a center contractive set of small enough size.
In the next section we consider a particular family of random graphs and we provide a deeper study on this trade-off.
\subsection{Geometric Random Graphs}

In the following we consider a random graph model that captures certain features of real world networks.
Consider a Poisson point process with uniform intensity $n\in \ZZ_+$ on the unit square $[0,1]^2$ and denote its point set by $V$.
%The intensity function is uniform equal to $n\in \ZZ_+$, in particular $\EE(|V|)=n$.
This process exhibits two key aspects.
\begin{itemize}
	\item[$(a)$] For every Lebesgue measurable set $\Omega\subseteq [0,1]^2$, we have that $|V\cap \Omega|\sim \text{Poisson}(n \nu(\Omega))$, where $\nu$ is the Lebesgue measure in $\RR^2$. 
	\item[$(b)$] For every pair of disjoint Lebesgue measurable sets $\Omega_1,\Omega_2\subseteq [0,1]^2$, we have that $|V\cap \Omega_1|$ is independent from $|V\cap \Omega_2|$.
\end{itemize}
In particular, from the first property we have that $\EE(|V|)=n$, that is, the process generates $n$ points on expectation. 
Consider the random graph $(V_n,E_{n,\rho})$ where $V_n$ is the point set of the above Poisson point process, and 
\begin{equation*}
E_{n,\rho}=\Big\{\{u,v\}\subseteq V: u\in B_{\infty}(v,\rho)\text{ and }u\ne v\Big\},
\end{equation*}
where $\rho$ is a {\it radius} value in $[0,1]$, and $B_1(v,\rho)$ is the $\ell_{\infty}$ ball centered at $v$ and with radius $\rho$.
That is, every node in $V$ is connected to every other node that is within a radius $\rho$.
We denote by $\poi(n,\rho)$ such random graph, and we denote by $\PP_{n,\rho}$ the probability distribution of the random variable $\poi(n,\rho)$. 
The notation extends to the expectation and higher moments as well.\\
\vvcom{Incluir un mono de ejemplo, debe haber algun generador online o en python seguro}
%\vvcom{podemos trabajar con $\ell_1$ or $\ell_2$, difiere por un factor constante solamente}

\noindent{\it Center Contractiveness.} It is well known that geometric random graphs exhibit a {\it threshold behavior} regarding its connectivity~\cite{}.
More specifically, there exists a constant value $\theta\in [0,1]$ such that for every $\varepsilon>0$ and $\rho\ge \theta(1+\varepsilon)\sqrt{\log(n)/n}$, the graph $\poi(n,\rho)$ is connected with high probability~\footnote{We say that an event holds {\it with high probability} if it occurs with a probability of at least $1-1/n^k$ where $k\ge 2$.}. 
And on the other hand, for every $\varepsilon>0$ and $\rho\le \theta(1-\varepsilon)\sqrt{\log(n)/n}$, the graph $\poi(n,\rho)$ is not connected with high probability.
In the following we assume that $\rho>\rho_n=\theta\sqrt{\log(n)/n}$.
The main result of this section is the following.

\begin{theorem}
\label{thm:geometric}
For every $\color{red} n\ge n_0$ and for every radius $\rho\in (\rho_n,1)$, the random geometric graph $\poi(n,\rho)$ is $({\color{red}1-\gamma_n},{\color{red} 4n\rho^2f^2 })$-center contractive {\color{blue} with high probability}.
\end{theorem}
\vvcom{Creo que hay un tradeoff entre la probabilidad de un nodo de ser cubierto por un center contractive, y el numero total de nodos que son cubiertos por algun center contractive. Idealmente, queremos que casi todos sean cubiertos con alta probabilidad, pero eso dependera de cuan fuertes son las cotas en los lemas que vienen.}
That is, with high probability, 
the fraction of nodes in the random graph that belong to a center contractive set is equal to ${\color{red}1-\gamma_n}$.
%every node in the random graph belongs to a center contractive set.
And furthermore, with high probability, the size of the center contractive set is ${\color{red} 4n\rho^2f^2}$.
Together with Theorem~\ref{thm:alternatives} we get the following corollary.
\vvcom{$g(n)$ seria algo asi como $O(\log(n))$}
\begin{corollary}
Consider $n\ge n_0$ and a radius $\rho\in (\rho_n,1)$.
Then, {\color{blue} with high probability}, a $1-\gamma_n$ fraction of the nodes in the the random graph $\poi(n,\rho)$ belongs to a set of at size at most $\color{red}4n\rho^2f^2$ that is robustly sustainable in the long-run.
\end{corollary}
\vvcom{incluir aca una breve discusion sobre como se compara este resultado con el Theorem 2 de Xandri}
\noindent In the rest of section we prove Theorem~\ref{thm:geometric}.

\subsection{Contractiveness in Geometric Random Graphs: Proof of Theorem~\ref{thm:geometric}}

The first property we state is that whenever a graph is sampled according to the geometric model, for every subset of nodes the shifting parameter is vanishing with high probability.
In what follows, given $n\in \ZZ_+$ and $\rho\in (\rho_n,1)$, for a node $u\in V_n$ we denote by $C_u=E(\poi(n,\rho))\cap B_1(u,\rho f)$ the set of nodes connected to $u$ within a radius of $\rho f$, with $f\in (0,1)$.
Recall that $\EE(C_u)=4n\rho^2f^2$.
In what follows, we will study the probability of a sequence of events, and for clarity we define them before going the precise statements of our technical lemmas.
Consider $n\in \ZZ_+$ and a radius $\rho\in (\rho_n,1)$. 
For every $u\in V_n$, consider the following events.
%Furthermore, it is zero on expectation
\begin{itemize}
	\item[$(a)$]  $\Omega_u$ is the event in which $C_u$ is center contractive.
	\item[$(b)$] $\Gamma_u$ be the event in which $|C_u|=\Theta(n\rho^2f^2)$.
	\item[$(c)$] $\calH_u$ is the event in which $\xi_{C_u}\le {\color{red}1/q(n)}$.
	\item[$(d)$] $\mathcal{K}_u$ is the event in which $C_u$ is $(1/2+1/q(n))$-cohesive.
	\item[$(e)$] $\calF_u$ is the event in which for every $v\in \nsf(u)\setminus C_u$ we have that 
	\begin{equation*}
	|\Theta_{v,C_u-u}-\Theta_{u,C_u-u}|\le \frac{1}{{\color{red}q(n)}}O(1-\Theta_{u,C_u-u}).
	\end{equation*}
	\item[$(f)$] $\mathcal{J}_u$ is the event in which $\Theta_{u,C_u-u}=O(\rho^2f^2)$.
\end{itemize}


In order to prove the theorem above, we follow a similar strategy used by Chandrasekhar et al~\cite{} to bound the share of nodes belonging to {\it clans}.
The following technical result is at the core of the argument to show the previous result, and can be of independent interest.

\begin{theorem}
\label{thm:cohesive}
Let $\color{red} n\ge n_3$, a radius $\rho\in (\rho_n,1)$ and $u\in V_n$ such that $|C_u|=\Theta(n\rho^2f^2)$.
%and $\omega\in (0,1)$.
Then, the set $C_u$ is $\omega$-cohesive with probability at least $\color{blue} f(\rho,\omega,n)$.
%$(1/2+1/n)$-cohesive.
% and $|C_u|\le 8n\rho^2f^2$. 
\end{theorem}

Before proving this result we show how to prove Theorem~\ref{thm:geometric}
We need a few more technical lemmas that we state next.
We leave the proof of Theorem~\ref{thm:cohesive} in Section~\ref{sec:cohesive-thm}.
\begin{lemma}
\label{lem:geo1}
%Let $n\in \ZZ_+$ and a radius $\rho\in (\rho_n,1)$.
%Then, for every proper $C\subseteq V$ and for every $\varepsilon>0$, the following holds.
%\begin{itemize}
%	\item[$(a)$] 
%For every $\color{red} n\ge n_1$ and for every radius $\rho\in (\rho_n,1)$, it holds that for every $u\in V_n$, 
Let $\color{red} n\ge n_3$, a radius $\rho\in (\rho_n,1)$ and $u\in V_n$ such that $|C_u|=\Theta(n\rho^2f^2)$.
Then, $\xi_{C_u}\le 1/{\color{red}q(n)}$ with probability at least $1-\kappa_n$.
%that is, $\PP(\calH_u|\Gamma_u)\ge 1-\kappa_n$.
%\begin{equation*}\PP_{n,\rho}\Big(\xi_{C_u}\ge 1/n\Big)\le {\color{red} \kappa_n}.\end{equation*}
%	\item[$(b)$] For every $u\in C$, we have $\EE_{n,\rho}(\xi_C(u))=0$.
%\end{itemize}
\end{lemma}

\begin{proof}
That is, we show next that $\PP(\calH_u|\Gamma_u)\ge 1-\kappa_n$.
In the following, we show that $\PP_{n,\rho}(\calF_u\cap \mathcal{J}_u|\Gamma_u)\ge 1-\nu_n$.
\vvcom{completar aca. La idea es ver que si $C_u$ tiene un radio suficientemente grande entonces el hitting probability the $C$ es harto mas grande que la diferencia entre los hitting partiendo de un punto $u$ o un punto $v$ vecino a $u$. Idealmente nos basta con $\rho=\Theta(\rho_n)$, pero si no lo agrandamos un poco. La diea tambien es que el grafo al ser random, no deberia cambiar mucho la hitting prob si partimos desde $v$ o desde $u$.}

\end{proof}

%\begin{lemma}
%\label{lem:geo2}
%For every $\color{red} n\ge n_2$, for every radius $\rho\in (\rho_n,1)$ and for every $u\in V_n$, we have that $u$ is $(1/2+1/n,\color{red} 8n\rho^2f^2)$-cohesive with probability at least ${\color{red}1-\beta_n}$.
%\end{lemma}


\begin{lemma}
\label{lem:geo2}
Let $\color{red} n\ge n_3$, a radius $\rho\in (\rho_n,1)$ and $u\in V_n$ such that $|C_u|=\Theta(n\rho^2f^2)$.
Then, the set $C_u$ is center contractive with probability at least ${\color{red}1-\beta_n}$, that is, $\PP_{n,\rho}(\Omega_u|\Gamma_u)\ge 1-\beta_n$.
%$(1/2+1/n)$-cohesive.
% and $|C_u|\le 8n\rho^2f^2$. 
\end{lemma}

\begin{proof}[Proof of Lemma~\ref{lem:geo2}]
%By Lemma~\ref{lem:geo3}, we have that for every node $u\in V_n$, the set $C_u$ is $(1/2+n^{-2})$-cohesive with probability at least $1-\alpha_n$.
%By Lemma~\ref{lem:geo1} and the union bound, we have that for a fixed $C\subseteq V$,
%\begin{equation*}
%\PP_{n,\rho}\Big(\xi_C>1/n\Big)\le \sum_{u\in C}\PP_{n,\rho}\Big(\xi_C(u)\ge 1/n\Big)\le |C|\kappa_n.
%\end{equation*}
%Fix a node $u\in V_n$ and let $\Omega_u$ be the event in which $C_u$ is $(1/2+\xi_{C_u})$-cohesive and $|C_u|\le 8n\rho^2f^2$.
%In particular, by conditioning on the event in which $\xi_C\le 1/n$ we get 
%\begin{equation*}
%\PP(\Omega_n)\ge (1-|C_u|\kappa_n)(1-\beta_n).
%\end{equation*}
%In this inequality we used that if $\xi_C\le 1/n$, then $(1/2+1/n)$-cohesiveness implies $(1/2+\xi_C)$-cohesiveness.
%By conditioning and using Lemma~\ref{lem:geo1} we have that $\PP_{n,\rho}(\calH_u)\ge 1-\Theta(n\rho^2f^2)\kappa_n$.
%\begin{equation*}
%\PP_{n,\rho}\Big(\xi_C>1/n\Big|\Gamma_u\Big)\le \sum_{u\in C}\PP_{n,\rho}\Big(\xi_C(u)\ge 1/n\Big)\le |C|\kappa_n.
%\end{equation*}
In order to conclude, thanks to Proposition~\ref{prop:coh-cont} it is enough to lower bound the probability that $C_u$ is $(1/2+1/q(n))$-cohesive when $|C_u|=\Theta(n\rho^2f^2)$.
%Therefore, we get overall that
%\begin{equation*}
%\PP(\Omega_u\cap \Gamma_u)\ge  (1-8n\rho^2f^2\kappa_n)\Big({\color{red}1-e^{-2n\rho^2f^2}}\Big).
%\end{equation*}
This comes from the following fact: When $\xi_{C_u}\le 1/q(n)$, we have that $(1/2+1/q(n))$-cohesiveness implies $(1/2+\xi_{C_u})$-cohesiveness.
In particular, this implies that 
\begin{align*}
\PP_{n,\rho}(\Omega_u|\Gamma_u)&\ge \PP_{n,\rho}(\Omega_u\cap \calH_u|\Gamma_u)\cdot \PP_{n,\rho}(\calH_u|\Gamma_u)\ge \PP_{n,\rho}(\mathcal{K}_u\cap \calH_u|\Gamma_u)\cdot \PP_{n,\rho}(\calH_u|\Gamma_u).
\end{align*}
By Theorem~\ref{thm:cohesive} we have that with probability at least $f(\rho,1/2+1/q(n),n)$ the set $C_u$ is $(1/2+1/1(n))$-cohesive when $|C_u|=\Theta(n\rho^2f^2)$, that is, $\PP_{n,\rho}(\mathcal{K}_u|\Gamma_u)\ge f(\rho,1/2+1/q(n),n)$.
Together with the union bound and Lemma~\ref{lem:geo1} we get that
\begin{align*}
\PP_{n,\rho}(\mathcal{K}_u\cap \calH_u|\Gamma_u)&\ge 1-\PP_{n,\rho}(\mathcal{K}^c_u|\Gamma_u)-\PP_{n,\rho}(\mathcal{H}^c_u|\Gamma_u)\ge f(\rho,1/2+1/q(n),n)-\kappa_n.
\end{align*}
Finally, overall we get that 
\begin{equation*}
\PP_{n,\rho}(\Omega_u|\Gamma_u)\ge \Big(f(\rho,1/2+1/q(n),n)-\kappa_n\Big)(1-\kappa_n)={\color{red} 1-\beta_n}.\qedhere
\end{equation*}
%To lower bound the probability that $C_u$ is $(1/2+1/n)$-cohesive we consider the following events.
%Let $\calF_u$ the event in which for every $v\in \nsf(u)\setminus C_u$ we have that $|\Theta_{v,C-u}-\Theta_{u,C-u}|\le \frac{1}{n}O(1-\Theta_{u,C-u})$.
%We lower bound the probability of that event using Lemma~\ref{lem:geo2}.
\end{proof}

\begin{lemma}
\label{lem:geo3}
Let $\color{red} n\ge n_3$, a radius $\rho\in (\rho_n,1)$ and $u\in V_n$.
Then, we have that $2n\rho^2f^2\le |C_u|\le 6n\rho^2f^2$ with probability at least ${\color{red}1-2e^{-n\rho^2f^2/3}}$.
% and $|C_u|\le 8n\rho^2f^2$. 
\end{lemma}

\begin{proof}
%Recall that $|C_u|\sim \text{Poisson}(4n\rho^2f^2)$.
That is, we show next that $\PP_{n,\rho}(\Gamma_u)\ge {\color{red}1-2e^{-n\rho^2f^2/3}}$.
In our analysis we use the following concentration bound for a Poisson random variable. 
The proof of this fact can be found in the Appendix.
\begin{claim}
\label{claim:concentration-poisson}
Consider a random variable $X\sim \text{Poisson}(\lambda)$ with $\lambda>0$.
Then, $\PP(|X-\lambda|>\lambda/2)\le 2e^{-\lambda/12}$.
\end{claim}
\noindent Using the claim for the random variable $|C_u|\sim \text{Poisson}(4n\rho^2f^2)$ we have that 
\begin{equation*}\PP_{n,\rho}\Big(|C_u-4n\rho^2f^2|>2n\rho^2f^2\Big)\le 2e^{-n\rho^2f^2/3},\end{equation*}
which concludes the proof.
%Therefore, every node $u\in V_n$ is $(1/2+1/n,8n\rho^2f^2)$-cohesive with probability at least 
%\begin{equation*}
%(1-\alpha_n)\Big(1-2e^{-n\rho^2f^2}\Big).
%\end{equation*}
\end{proof}

\begin{proof}[Proof of Theorem~\ref{thm:geometric}]

Fix a node $u\in V_n$. 
%and let $\Omega_u$ be the event in which $C_u$ is center contractive and let $\Gamma_u$ be the event in which $|C_u|=\Theta(n\rho^2f^2)$.
In order to conclude the result it is enough to show that $\PP_{n,\rho}(\Omega_u\cap \Gamma_u)\ge {\color{red}1-\gamma_n}$.
%, since in that case $C_u$ is center contractive by Proposition~\ref{prop:coh-cont}.
The theorem follows by conditioning and using Lemmas~\ref{lem:geo2} and~\ref{lem:geo3},
\begin{equation*}
\PP_{n,\rho}(\Omega_u\cap \Gamma_u)=\PP_{n,\rho}(\Omega_u|\Gamma_u)\cdot \PP_{n,\rho}(\Gamma_u)\ge ({\color{red}1-\beta_n})\Big({\color{red}1-2e^{-n\rho^2f^2}}\Big)=1-\gamma_n.\qedhere
\end{equation*}
%That concludes the proof.
\end{proof}

\subsection{Cohesiveness in Geometric Random Graphs: Proof of Theorem~\ref{thm:cohesive}}
\label{sec:cohesive-thm}

{\color{blue} Extender (pero a la vez simplificar) el analisis de Chandrasekhar et al entre pags 44-48, y tambien necesitamos estudiar cual es probabilidad de que efectivamente un nodo este en un conjunto 1/2-cohesive. Para eso hay que estudiar la cota inferior, idealmente tenemos probabilidad tendiendo a 1 simptoticamente, sino prob constante esta bien igual.}
%Observe that thanks to Proposition~\ref{prop:coh-cont}, in order to prove Theorem~\ref{thm:geometric} it is enough to show that for every $n\in \ZZ_+$ and $\rho\in (\rho_n,1)$, with high probability the graph $\poi(n,\rho)$ is $(1-o(1),1/2+\xi_G)$
%\subsection{Power Law Random Graphs}

\section{Cohesiveness and Spectral Gap}
%Recall that $\Omega^C$ is a linear function in $(\gamma_0,\beta)$, and therefore $f_u$ is a quadratic function on $\beta$.
It will be convenient to express the cost function in matrix form.
For every $u\in V$, consider the matrix $\asf_u\in \RR^{V\times V}$ such that $\asf_u(u,v)=-1$ when $v\in \nsf(u)$ and zero otherwise; and let $\dsf\in \RR^{V\times V}$ the diagonal matrix such that $\dsf(u,u)=\degsf(u)$ for every $u\in V$.
Then, for every $y\in \RR^V$ we have that 
\begin{equation*}
\cost_u(y)=\frac{1}{2}y^{\top}\lsf_u y,
\end{equation*}
where $\lsf_u=\dsf-\asf_u$ and we call this matrix the {\it local Laplacian} for node $u\in V$.
In particular, the matrix $\lsf=\sum_{u\in V}\lsf_u$ is known as the {\it graph Laplacian} of $G$.\\

\noindent{\it Spectral Gap and Cheeger's Inequality.} 
\begin{example}

\end{example}

%\section{Convergence Times}
\begin{comment}
\section{Bounding the discounted cost}

By splitting the discounted cost in the mixing time, we obtain the following bound.

\begin{proposition}
{\color{red} elaborar aca}
\begin{equation}
\sum_{t= 0}^{\infty}\delta^t\cost_u(\gamma_t)\le \sum_{t=0}^{\tmix(\mathcal{G})}\delta^t\cost_u(\gamma_t)+\frac{\delta^{\tmix(\mathcal{G})}}{1-\delta}(\cost_u(\gamma_{\infty})+f(\beta)).
\end{equation}
\end{proposition}
\end{comment}
%{\it Graphs.} Graphs $G = (V, E)$ considered in this paper may be directed or
%undirected; typically we assume $|V| = n$ and $|E| = m$, though if there is
%scope for confusion we use $|V|$ or $|E|$ explicitly. For undirected graphs,
%for a node $v \in V$, we denote by $N(v)$ its \emph{neighbourhood}, that is $N(v) =
%\{ w ~|~ \{v, w \} \in E \}$, and its degree by $\deg(v) := |N(v)|$. In the
%case of directed graphs, we denote $N^+(v) := \{ w ~|~ (v, w) \in E \}$ its
%\emph{out-neighbourhood} and by $\deg^+(v) := |N^+(v)|$ its \emph{out-degree}.
%Similarly, $N^-(v) := \{ u ~|~ (u, v) \in E \}$ denotes its
%\emph{in-neighbourhood} and $\deg^-(v) := |N^-(v)|$ its \emph{in-degree}.
%%Furthermore, $\davg$ denotes the average degree, i.e, $\sum_ {v\in V}\deg(v)/n$.  
%%Whenever there is scope for confusion, we use the notations
%%$\deg_G(u)$, $N_G(v)$, $\davg(G)$, \etc to emphasise that the terms are with
%%respect to graph $G$. \medskip
%
%\noindent {\it Random walks in graphs.} 
%\vvcom{tengo que seguir modificando aca, adaptar a la nocion de RW que necesitamos aca}
%A discrete-time lazy random walk
%$(X_t)_{t\ge 0}$ on a graph $G=(V,E)$ is defined by a Markov chain with state
%space $V$ and transition matrix $P=(p(u,v))_{u,v\in V}$ defined as follows: 
% For every $u \in V$, $p(u,u) = 1/2$ ({\it Laziness}).
%In the undirected setting, for every $v \in N(u)$, $p(u,v) = 1/(2
%		\deg(u))$. 
%%In the directed setting, for every $v \in N^+(u)$, $p(u,v) =1/(2 \deg^+(u))$. 
%%%
%The transition probabilities can be expressed in matrix form as $P=(I+
%\mathcal{D}^{-1}A)/2$, where $A$ is the adjacency matrix of $G$, $\mathcal{D}$
%is the diagonal matrix of node degrees (only out-degrees if $G$ is directed),
%and $I$ is the identity. Let $p^t(u, \cdot)$ denote the distribution over nodes
%of a random walk at time step $t$ with $X_0 = u$. For the most part, we will
%consider (strongly) connected graphs. Together with laziness, this ensures that
%the stationary distribution of the random walk, denoted by $\pi$, is unique and
%given by $\pi P=\pi$. In the undirected case, the form of the stationary
%distribution is particularly simple, $\pi(u) = \deg(u)/(2|E|)$; furthermore, the
%random walk is reversible, that is $\pi(u)  p(u,v) = \pi(v) p(v,u)$. As before,
%$\pi_G$ is used to emphasise that the stationary distribution is respect to
%graph $G$. \medskip

%\noindent {\it Mixing time.} To measure how far $p^t(u,\cdot)$ is from the
%stationary distribution we consider the {\it total variation distance}; for
%distributions $\mu,\nu$  over sample space $\Omega$ the total variation
%distance is $\tv(\mu,\nu)=\frac{1}{2}\displaystyle\sum_{x\in \Omega}|\mu(x)-\nu(x)|$.  The
%\emph{mixing time} of the random walk is defined as
%\[
%\tmix := \max_{u \in V} \min \{ t \geq 1 ~|~ \tv(p^t(u, \cdot),\pi) \leq e^{-1} \}.
%\]
%%
%Although the choice of $e^{-1}$ is arbitrary, it is known that after
%$\tmix\log(1/\varepsilon)$ steps, the total variation distance is at most
%$\epsilon$. 

\begin{comment}
\section{Commentarios/Intuiciones}
\begin{conjecture}
For every graph $G$ satisfying certain conditions, there exists a sustainable coordination.
\end{conjecture}

\vvcom{si $C=V$ siempre tenemos un coordination sustainable cierto? probablemente sea interesante probar que existe un $C$ "pequenho" que sustenta coordination} 
\begin{intuition}
For every coordination, the initial opinions can not be too far.
\end{intuition}

\begin{intuition}
There exists a trade-off between how well connected is a node $v\in C$ with the regular nodes, and how far the posted opinion could be from its natural (DeGroot) opinion.
\end{intuition}

\begin{intuition}
On the one hand, say we have $C$ of cardinality two. If one of the nodes is very well connected, could incur a high cost by not listening to the rest of the network. On the other hand, if we post an opinion similar to the one followed by the very well connected node, the cost incured will be low, and even more, the variance in the neighbourhood might decrease, which could help to sustain the coordination. 
\end{intuition}

\begin{intuition}
Let $\phi(S)=\partial S/|E(S)|$, where $\partial S$ is the cut of $S$ and $E(S)$ is the edges connecting nodes both in $S$.
If $\phi(S)$ is small, the set $S$ is badly connected with the exterior (in proportion), and then we can expect to sustain a coordination, in certain regime. 
\end{intuition}

\begin{intuition}
Now suppose that $\phi(S)=O(1)$ and $E(S)$ is small (poorly connected within $S$), then we can expect that sustaining a coordination is harder. With decent probability, there will be a node with incentives to deviate. 
\end{intuition}

\begin{intuition}
We say that a node is problematic if it maximizes the connectivity to the rest of the world in comparison to the connectivity with the coalition (in ratio). It could be that this type of nodes dominate the rest in terms of deviation: if that type of node has no incentive to deviate, neither the rest.
\end{intuition}
\section{{\color{red} Otras ideas a considerar:}}

\begin{itemize}
\item Podriamos definifir un operador que minimize la dispersión del bloque como,

\begin{equation}
\text{var}_C(\beta,\theta)=(1-\eta)(\beta-\theta(C))^2+ \frac{\eta}{\delta_c(C)} \sum_{w\in N_c(C)}(\beta-\theta(w))^2,
\end{equation}
donde $\theta(w)$ es la opinión de los miembros de $C$ según DeGroot.

	\item Seria interesante poder cuantificar cuanto gana una coordinacion versus no coordinarse, eso daria una especie de {\it price of coordination/collusion}.
	\item Ver los otros papers Acemoglu/experimentos para sacar mas ideas sobre que otra caracteristica/comportamiento podria ser interesante de medir o estudiar.
\end{itemize}
\end{comment}
\bibliographystyle{plain}
\bibliography{bibliography}

\section{Appendix}

\begin{proof}[Proof of Lemma~\ref{lem:limit-thm}]

\end{proof}

\begin{proof}[Proof of Claim~\ref{claim:concentration-poisson}]

\end{proof}

\end{document}
