\documentclass[11pt]{article}

\usepackage{amsmath,amssymb}

\usepackage[utf8]{inputenc}

\newcommand{\R}{\mathbb{R}}


\begin{document}


\section*{Viernes 5 Abril}

\begin{itemize}
	\item Estamos considerando el modelo de DeGroot modificado a lo Kleinberg con tolerancia.
	\item A este modelo de incluimos un umbral de tolerancia, y si la opinión de mi vecino está más alejada de la mía que ese umbral, lo multiplico por cero. (Tolerancia limitada)
	\item Comparamos, corremos las dinámicas de ambos modelos y queremos medir las distancias entre los estados estacionarios de ambas dinámicas, y medimos las distancias:
	\begin{itemize}
	\item Si la distancia es grande, tiene poca resistencia a polarizarse.
	\item Si la distancia es chica, tiene alta resistencia a polarizarse.
	\item Por ejemplo, la distancia puede ser:
	\begin{align*}
		\vert \vert \Pi_L - \Pi_{\infty}\vert \vert _p
	\end{align*}
	\item Idea de modelo:
	\begin{align*}
		G(V,E)\\
		x_i(t)\in \R^V \\
		x_i(t+1) = \alpha x_i(t) + (1-\alpha) \frac{1}{\vert {\cal N}_t^{\varepsilon}(i) \vert} \sum_{j\in {\cal N}_t^{\varepsilon}(i)} x_j(t) 
	\end{align*}
	donde,
	\begin{align*}
		{\cal N}_t^{\varepsilon}(i)=\{j\in {\cal N}(i): \vert x_i(t)-x_j(t)\vert <\varepsilon   \}
	\end{align*}
\end{itemize}

\end{itemize}
\section*{Viernes 26 de Abril}

\begin{itemize}
	\item Consideraremos un cartel a un grupo de nodos conectados que puede coordinarse en emitir una opinión $z^*$.
	\item Cada miembro del cartel, actualizara su opinión, sólo considerando a los  otros miembros del cartel. 
	\item El resto del grafo actualiza de la manera usual.
	\item Preguntas ¿Cuál es la opinión óptima que elige el cartel? ¿Cuál es el costo de desviarse?. Para sustentar la opinión colusiva, ¿ponderaremos la opinión $z^*$ por el costo de desviarse o la conectividad de los miembros?
\end{itemize}
\section*{Viernes 3 Mayo}
\end{document}
